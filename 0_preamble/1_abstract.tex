%% ------------------------------ Dissertation Abstract ---------------------------------- %%
%% Dissertation Title: A Data-Parallel Monte-Carlo Framework for Large-Scale PRA using Probabilistic Circuits
\begin{abstract}
Probabilistic risk assessment (PRA) for nuclear systems typically requires enumerating minimal cut sets or essential prime implicants to capture all possible sequences of component failures, an approach that grows exponentially more complicated for large models. Existing methods combat this complexity by imposing structural restrictions on logic gates, applying rare-event approximations, or using bounding techniques like the min-cut upper bound. In this dissertation, we propose a data-parallel Monte Carlo framework that directly estimates probability distributions over entire Boolean spaces, circumventing many of the constraints faced by standard cut set analyses. By sampling from component-level input distributions rather than symbolically evaluating each logic configuration, our approach quantifies both success and failure events in one pass. Our open-source framework, mcSCRAM, packs component states into integer bit-vectors and uses vectorized bitwise operations to achieve high throughput across modern GPUs and multi-core CPUs using SYCL. As an extension, we introduce Monte Carlo analogs for common cause failure analysis, sensitivity analysis using first-order importance measures, and a pilot implementation for importance sampling to deal with rare-events. In addition, we explore the constraints and opportunities for Boolean knowledge compilation under the relaxed Monte Carlo paradigm. Finally, we explore inverse problems, and partial-derivatives using bitwise operations, which provides a pathway towards future work on gradient-based updates and other learning-based tasks.

We verify our implementation against the Aralia dataset, comparing convergence and accuracy with the open source SCRAM tool. Preliminary results demonstrate that with the exception of rare-events, data-parallel Monte Carlo, which is primarily memory-bound, can sample a few hundred million events in under 300ms while fully saturating available resources, even on older-generation consumer GPUs. For fault trees with a hundreds of events, sampling achieves error margins comparable to exact solvers.

We anticipate future work on benchmarking the generic pressurized water reactor PRA model, in comparison with industry-standard tools (CAFTA, FTREX, SAPHIRE, XFTA), to provide additional insights into solver performance and bottlenecks as the PRA model scales. 
\end{abstract}