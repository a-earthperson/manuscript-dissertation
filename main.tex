% !TeX TS-program = pdflatex 

\documentclass[11pt,          % font size: 11pt or 12pt
               phd,           % degree:    ms or phd
               onehalfspacing % spacing: onehalfspacing or doublespacing
               ]{ncsuthesis}

%\documentclass[openany,oneside,titlepage,letterpaper]{book}

%%----------------------------------------------------------------------------%%
%%------------------------------ Import Packages -----------------------------%%
%%----------------------------------------------------------------------------%%
\usepackage{xcolor}
\usepackage{algorithm2e}
\usepackage{booktabs}  % professionally typeset tables
\usepackage{amsmath}
\usepackage{amsthm}
\usepackage{amssymb} 
\usepackage{amsfonts}
\usepackage{tikz}
\usepackage{tikz-qtree}
\usetikzlibrary{calc,positioning}
\usepackage{textcomp}  % better copyright sign, among other things
\usepackage{lscape}
\usepackage{longtable}
\usepackage{multirow}
\usepackage{subfig}    % composite figures
\usepackage{natbib}    % ability to use citet,citep
% \usepackage{fancyhdr}  % creates headers

\usepackage{siunitx} % For aligning numbers by decimal point

%%% For accessing system, OTF and TTF fonts
%%% (would have been loaded by polylossia anyway)
\usepackage{fontspec}
\usepackage{xunicode} %% loading this first to avoid clash with bidi/arabic

%%% For language switching -- like babel, but for xelatex
\usepackage{polyglossia}

\setmainlanguage{english}
\setotherlanguages{hindi,sanskrit} %% or other languages
\newfontfamily\devanagarifont[Script=Devanagari]{Noto Serif Devanagari}


%Citations should be of the form ``author year''  not ``author, year''
\bibpunct{(}{)}{;}{a}{}{,} % changes apalike bst into AMS format
 
%%----------------------------------------------------------------------------%%
%%---------------------------- Formatting Options ----------------------------%%
%%----------------------------------------------------------------------------%%
%%

%% -------------------------------------------------------------------------- %%
%% Disposition format -- any titles, headings, section titles
%%  These formatting commands affect all headings, titles, headings,
%%  so sizing commands should not be used here.
%%  Formatting options to consider are
%%     +  \sffamily - sans serif fonts.  Dispositions are often typeset in
%%                    sans serif, so this is a good option. 
%%     +  \rmfamily - serif fonts
%%     +  \bfseries - bold face
%\dispositionformat{\sffamily\bfseries}   % bold and sans serif
\dispositionformat{\bfseries}            % bold and serif

%% -------------------------------------------------------------------------- %%
%% Formatting for centered headings - Abstract, Dedication, etc. headings
%%  This is where one might put a sizing command.
%%  \MakeUppercase can be used to typeset all headings in uppercase.
\headingformat{\large\MakeUppercase}   % All letters uppercase
%\headingformat{\large}                % Not all uppercase
%\headingformat{\Large\scshape}        % Small Caps, used with serif fonts.

%% Typographers recommend using a normal inter-word space after
%% sentences. TeX's default is to add an wider space, but \frenchspacing
%% gives a normal spacing. Comment out the following line if you prefer wider spaces between sentences.
\frenchspacing

%% -------------------------------------------------------------------------- %%
%%  Optional packages
%%    A number of compatible packages to improve the look and feel of
%%    your document are available in the file optional.tex 
%%    (For example, hyperlinks, fancy chapter headings, and fonts)
%% To use these options, uncomment the next line and see optional.tex
%%  Optional Packages to consider.   These packages are compatible with
%%    ncsuthesis.  

%% -------------------------------------------------------------------------- %%
%% Fancy chapter headings
%%  available options: Sonny, Lenny, Glenn, Conny, Rejne, Bjarne
%\usepackage[Bjornstrup]{fncychap}
%\usepackage[Rejne]{fncychap}
%\usepackage[Sonny]{fncychap}
\newcommand{\headertitlefont}{%
    \fontsize{8}{12}\selectfont
}
\fancyhf{}
\fancyhead[L]{\headertitlefont \leftmark} %chapter
\fancyhead[R]{\headertitlefont \rightmark} %section
\fancyfoot[C]{\thepage} %footer
\pagestyle{fancy}     


%%----------------------------------------------------------------------------%%
%% Hyperref package creates PDF metadata and hyperlinks in Table of Contents
%%  and citations.  Based on feedback from the NCSU thesis editor, 
%%  the links are not visually distinct from normal text (i.e. no change
%%  in color or extra boxes).
\usepackage[
  pdfauthor={Arjun Earthperson},
  pdftitle={Change title in optional.tex},
  pdfcreator={pdftex},
  pdfsubject={NC State ETD Dissertation},
  pdfkeywords={change key words in optional.tex},
  colorlinks=true,
  linkcolor=black,
  citecolor=black,
  filecolor=black,
  urlcolor=black,
]{hyperref}
\usepackage[xindy,acronym,toc]{glossaries}


%% -------------------------------------------------------------------------- %%
%% Microtype - If you use pdfTeX to compile your thesis, you can use
%%              the microtype package to access advanced typographic
%%              features.  By default, using the microtype package enables
%%              character protrusion (placing glyphs a hair past the right 
%%              margin to make a visually straighter edge)
%%              and font expansion (adjusting font width slightly to get 
%%              more favorable justification).
%%              Using microtype should decrease the number of lines
%%              ending in hyphens.
\usepackage{microtype}


%%----------------------------------------------------------------------------%%
%% Fonts 

%% ETD guidelines don't specify the font.  You can enable the fonts
%%  by uncommenting the appropriate lines.  Using the default Computer 
%%  Modern fonts is *not* required.  A few common choices are below.
%%  See http://www.tug.dk/FontCatalogue/ for more options.

%% Serif Fonts -------------------------------------------------
%%  The four serif fonts listed here (Utopia, Palatino, Kerkis,
%%  and Times) all have math support.


%% Utopia
%\usepackage[T1]{fontenc}
%\usepackage[adobe-utopia]{mathdesign}

%% Palatino
% \usepackage[T1]{fontenc}
% \usepackage[sc]{mathpazo}
% \linespread{1.05}

%% Kerkis
% \usepackage[T1]{fontenc}
% \usepackage{kmath,kerkis}

%% Times
% \usepackage[T1]{fontenc}
% \usepackage{mathptmx}


%% Sans serif fonts -------------------------

%\usepackage[scaled]{helvet}  % Helvetica
%\usepackage[scaled]{berasans} % Bera Sans

\setcounter{tocdepth}{3}
\setcounter{secnumdepth}{4}

%% https://www.overleaf.com/learn/latex/Theorems_and_proofs
\newtheorem{theorem}{Theorem}[section]
\newtheorem{corollary}{Corollary}[theorem]
\newtheorem{lemma}[theorem]{Lemma}
\tikzset{
    grow'=down,
    level distance=42pt,
    % sibling distance=24pt,
    % sibling distance=8mm/#1,
    level/.style={sibling distance=3pt+44pt/(#1*#1*0.75)},
    edge from parent/.append style={
        draw,
        %thick,
        edge from parent path={
        (\tikzparentnode.west) |- ($(\tikzparentnode.west)!0.5!(\tikzchildnode.east)$) -| (\tikzchildnode.east)
        },
    },
    every node/.append style={
        anchor=center,
        rotate=90,
        draw=black,
        fill=white,
        thick,
        font=\footnotesize\bfseries,
        text centered,
        inner sep=0pt
    },
    var/.style={
        shape=circle,
        minimum height=16pt,
    },
    and/.style={
    and gate US,
    },
  not/.style={
    not gate US,
  },  
  or/.style={
    or gate US,
  },
  vot/.style={
    or gate US,
  }
}

%solve bug from fancyhdr in optional
%http://nw360.blogspot.com/2006/11/latex-headheight-is-too-small.html
%\setlength{\headheight}{26.94345pt} % corrected error in Overleaf
%\fancyhead[L]{\vspace{1mm}} % only puts chapter title in headers

%%----------------------------------------------------------------------------%%
%%---------------------------- Content Options -------------------------------%%
%%----------------------------------------------------------------------------%%
%% Size of committee: 3, 4, 5, or 6 -- this number includes the chair
\committeesize{4}

%% Members of committee
%%  Each of the following member commands takes an optional argument
%%   to specify their role on the committee.
%%  For co-chairs, use the commands:
%%      \cochairI{Doug Dodd}
%%      \cochairII{Chris Cox}
%%
\chair{Dr. Mihai A. Diaconeasa}
\memberII{Dr. Yousry Azmy}
\memberI{Dr. Aydin Aysu}
\memberIII{Dr. Nam Dinh}   % unnecessary if committeesize=3
% \memberIV{Member 4 name}    % unnecessary if committeesize=3, 4

%% Student writing thesis, \student{First Middle}{Last}
\student{Arjun}{Earthperson} % a full middle name

%% Degree program e.g. Marine, Earth, and Atmospheric Science
\program{Nuclear Engineering}

%%!!!!!! To Change Year !!!!!!%%
% If year of graduation is not same as current year (common for December graduates
% thanks to the Grad Schools odd graduation rules) go into ncsuthesis.cls and change 
% \the\year in the line:
% \newcommand{\ncsu@year}{\the\year}
% to the year of graduation. E.g.:
% \newcommand{\ncsu@year}{2020}

%% Thesis Title
%%  Keep in mind, according to ETD guidelines:
%%    +  Capitalize first letter of important words.
%%    +  Use inverted pyramid shape if title spans more than one line.
%%
%%  Note: To break the title onto multiple lines, use \break instead of \\.
% \thesistitle{A North Carolina State University Sample \LaTeX{} Thesis \break 
% with a Title So Long it Needs a Line Break}

% \thesistitle{A Sampling Scheme for Numerical Integration of Boolean Functions with Many Variables}

% \thesistitle{Finding Slopes in Large Boolean Spaces using Probabilistic Circuits}

% \thesistitle{Quantifying Risk by Simulating Probabilistic Circuits}

\thesistitle{A Data-Parallel Monte Carlo Framework for Large-Scale PRA using Probabilistic Circuits}
%% Degree year. Necessary if your degree year doesn't equal the current year.
\degreeyear{2025}

%% While your here make sure to change the PDF characteristics in optional.tex!!!

%%----------------------------------------------------------------------------%%
%%---------------------------- Personal Macros -------------------------------%%
%%----------------------------------------------------------------------------%%

%% A central location to add your favorite macros.

%% A few examples to get you started.
\newcommand{\uv}[1]{\ensuremath{\mathbf{\hat{#1}}}}
\newcommand{\bo}{\ensuremath{\mathbf{\Omega}}}
\newcommand{\eref}[1]{Eq.~\ref{#1}}
\newcommand{\fref}[1]{Fig.~\ref{#1}}
\newcommand{\tref}[1]{Table~\ref{#1}}

\theoremstyle{definition}
\newtheorem{definition}{Definition}[section]

%\makeglossaries

\newglossaryentry{cutset}
{
    name=Cut Set,
    description={A set of basic events (component failures) whose simultaneous occurrence causes the system to fail. In other words, the failure of all components in the set results in system failure.}
}

\newglossaryentry{pathset}
{
    name=Path Set,
    description={A set of components whose simultaneous functioning ensures system success. That is, if all components in the set are operational, the entire system functions properly.}
}

\newacronym{mcs}{MCS}{Minimal Cut Set}

\newglossaryentry{minimalcutset}
{
    name=Minimal Cut Set,
    description={A smallest possible Cut Set where the failure of all components causes system failure, and removing any component from the set would no longer result in system failure.},
    parent=cutset
}

\newacronym{mps}{MPS}{Minimal Path Set}

\newglossaryentry{minimalpathset}
{
    name=Minimal Path Set,
    description={A smallest possible Path Set where the functioning of all components ensures system success, and removing any component from the set would no longer ensure system success.},
    parent=pathset
}

\newglossaryentry{maximalpathset}
{
    name=Maximal Path Set,
    description={A Path Set that cannot be extended by adding more components; it includes all components that can be in a Path Set without redundancy. Adding any additional component does not create a new Path Set, as it would not change the system's operational status.}
}

\newacronym{pi}{PI}{Prime Implicant}

\newglossaryentry{primeimplicant}
{
    name=Prime Implicant,
    description={An implicant of a Boolean function that is as large as possible (in terms of combining variables) without containing redundant literals. It represents a minimal product term that cannot be combined further to simplify the function.},
    sort=Prime Implicant
}

\newacronym{epi}{EPI}{Essential Prime Implicant}

\newglossaryentry{essentialprimeimplicant}
{
    name=Essential Prime Implicant,
    description={A Prime Implicant that covers one or more minterms (true outputs) of a Boolean function that no other Prime Implicant covers. These are critical for the minimal expression of the function, as omitting them would alter the function's output.},
    parent=primeimplicant
}

\newacronym{et}{ET}{Event Tree}
\newglossaryentry{eventtree}{
    name=Event Tree,
    description={A logic diagram that begins with an initiating event or condition and progresses through a series of branches that represent expected system or operator performance that either succeeds or fails and arrives at either a successful or failed end state.},
}

\newacronym{PRA}{PRA}{Probabilistic Risk Assessment}
\newglossaryentry{pra}{
    name=Probabilistic Risk Assessment,
    description={A qualitative and quantitative assessment of the risk associated with plant operation and maintenance that is measured in terms of frequency of occurrence of risk metrics, such as release category frequency and its effects on the health of the public (also referred to as a PSA)},
}

% \newdualentry{et} % label
%   {ET}            % abbreviation
%   {Event Tree}  % long form
%   {A logic diagram that begins with an initiating event or condition and progresses through a series of branches that represent expected system or operator performance that either succeeds or fails and arrives at either a successful or failed end state} % description
  
% \glsaddall %  to list all entries <<<<<
% \usepackage[toc]{glossaries}
%\makeglossaries
%\glsaddall %  to list all entries <<<<<


%%---------------------------------------------------------------------------%%
\begin{document}
%%---------------------------------------------------------------------------%%
\frontmatter


\include{front}

\thesistableofcontents
\thesislistoftables
\thesislistoffigures
% \addcontentsline{toc}{chapter}{List of Algorithms}
% \listofalgorithms
% \thesisacronyms
% \thesisdefinitions

%%---------------------------------------------------------------------------%%
\mainmatter
% Chapters can remove or add
%\input{parts/0_intro/informal_overview}

\chapter{Introduction}
% The entire risk assessment enterprise can be summarized as the act of integrating/bounding uncertainties within what we know to be true. Putting scaffolding around the unknowns. Using the few knowns to better structure the unknowns. To give form to the unknown.
\section{Background \& Motivation}
Probabilistic risk assessment (PRA) aims to quantitatively evaluate the likelihood and severity of adverse events in safety-critical industries. Driven by seminal works such as WASH-1400 and subsequent regulatory guidance, PRA now serves as a cornerstone of risk-informed decision-making in nuclear engineering. A canonical feature of PRA is its reliance on  Boolean logic structures (fault trees and event trees) that characterize sequences of component failures and human actions leading to top-level undesirable outcomes. While such structures ensure thoroughness, the computational complexity of enumerating all failure paths grows exponentially in the number of components. Even moderate-scale reactor models may involve tens of thousands of basic events, rendering naive calculation of end-state probabilities intractable.

Over decades, analysts have adopted a series of approximations and bounding schemes to handle this combinatorial explosion. Strategies include rare-event approximations (which assume minimal overlap between failure sets), min-cut upper bounds (which treat all minimal cut sets as mutually exclusive), and restrictions on gate types to keep expansions manageable. Tools such as CAFTA, FTREX, SAPHIRE, SCRAM, and XFTA implement these methods and remain widely used in industry. Nonetheless, these approximations can lead to conservative estimations.

In recent years, the continued growth of computing power has encouraged reassessment of how PRA calculations can be modernized. Specifically, massively parallel hardware (e.g., GPUs and multi-core CPUs) has prompted the exploration of data-parallel methods. Monte Carlo sampling is a natural fit for parallelization: since each sample is independent, thousands or millions of system-state draws can be processed simultaneously to build empirical estimates of key probabilities. Straightforward sampling from component failures (rather than enumerating complex Boolean expansions) offers flexibility in modeling dependencies and higher-order correlations. The overarching purpose of this dissertation is to develop a data-parallel Monte Carlo framework for large-scale nuclear PRA, grounded in a GPU-friendly integer bit-packing approach and extended to advanced sensitivity analyses using partial derivatives via Shannon decomposition.
\section{Scope}
This work reexamines how to efficiently compute the failure probability of a large Boolean system while capturing a wide array of gate structures, potential dependencies, and partial derivatives for sensitivity. We concentrate on the following core questions:

\begin{itemize}
   \item How can large-scale PRA models be quantified without explicit minimal cut set enumeration or strict reliance on model simplifications?
   \item Which data structures and numerical techniques allow us to exploit parallel hardware such as GPUs, multi-core CPUs, and field-programmable gate arrays (FPGAs)?
   \item What are the current limitations of Monte Carlo (e.g., rare-event estimation and common-cause failure sampling), and how might variance reduction or more sophisticated sampling schemes mitigate them?
\end{itemize}
While our emphasis centers on nuclear applications, the proposed techniques and software are equally suitable for other industries that manage complex risk scenarios (e.g., aerospace, chemical processing, or automotive safety). The dissertation does not attempt to unify every advanced PRA feature (e.g., dynamic simulations or large correlated uncertainties), but it lays the foundation for an extensible data-parallel approach that can incorporate such features in the future.

\clearpage
\section{Outline and Contributions}
The primary technical contributions of this dissertation can be summarized as follows:

\begin{enumerate}
\item \textbf{Data-Parallel Monte Carlo for Boolean Systems:}  
We introduce a new framework that estimates probabilities for all \emph{success} and \emph{failure} states in a single run of Monte Carlo sampling. By representing each random system state in a bit-packed data structure, we achieve high-throughput simulations where Boolean operators (AND, OR, \(k/n\), etc.) map naturally to bitwise operations on GPUs or multi-core CPUs.

\item \textbf{Integration with Probabilistic Circuits:}  
To unify event/fault tree logic with more flexible gate structures, we embed the model in a \emph{probabilistic circuit} representation. This perspective enables node-level factorization and sum mixtures, opening doors to advanced decomposition-based analyses while retaining parallel-friendly evaluation.

\item \textbf{Sampling Techniques for Partial-Derivatives:}  
We develop a bitwise algorithm to approximate partial derivatives of the system’s failure probability with respect to individual or clustered component reliabilities. By evaluating logical expressions under complementary assignments (as guided by the Shannon expansion), these derivatives can be computed in the same Monte Carlo pass. This capability facilitates advanced sensitivity and importance ranking in large models. It also opens a path towards integrating model evaluation with learning-based tasks.

\item \textbf{Benchmarking Against Industry Tools:}  
Through a series of case studies—most notably, the generic pressurized water reactor (PWR) reference model—we compare our approach with standard PRA tools (CAFTA, FTREX, SAPHIRE, SCRAM, XFTA). Results indicate that at comparable accuracy, our framework can surpass existing methods by orders of magnitude in runtime performance. We discuss how discrepancies in extremely low-probability events should be carefully monitored via convergence diagnostics.

\item \textbf{Prototype Implementation:}  
We present an open-source reference implementation named Canopy, built using the SYCL programming model. The code is portable across a variety of parallel architectures, including consumer GPUs and specialized accelerators. We provide usage examples and discuss future directions, such as unifying the approach with importance sampling to better handle rare events and building correlated sampling routines amenable to common-cause failure modeling.
\end{enumerate}

\section{Software Implementations}
\section{Related Publications}
\section{Organization of the Dissertation}
% The remainder of this dissertation is organized into multiple parts covering theoretical foundations, methodological frameworks, practical results, and future directions:

% \begin{itemize}
% \item \textbf{Part I: Foundations}  
%    – Provides an overview of rich Boolean representations (event trees, fault trees) in PRA, introduces probabilistic circuits, and reviews key Monte Carlo sampling principles relevant to reliability estimation.

% \item \textbf{Part II: Proposed Methods}  
%    – Details the data-parallel Monte Carlo approach, discussing bit-packed state encoding, gate-by-gate evaluation, and Shannon-decomposition-based partial derivatives.

% \item \textbf{Part III: Implementation and Results}  
%    – Documents the Canopy software design, GPU kernel formulations, and parallel performance optimizations. Presents numerical benchmarks against large PWR fault-tree models and comparisons with standard PRA codes. Highlights areas needing specialized variance reduction for rare events.

% \item \textbf{Part IV: Extensions and Conclusion}  
%    – Explores potential enhancements, such as correlated sampling for common-cause failures and ways to incorporate Bayesian or machine-learning-based parameter updates. Summarizes major findings and discusses the prospective impact of data-parallel quantification on future PRA methodologies.

% \end{itemize}

% \part{Foundations}

\chapter{Towards Parameter Fitting}
\label{sec:parametric_learning_pra_model}

PRAs invariably involve uncertainty. When explicitly modeled, these uncertainties can be updated or inferred from evidence, engineering judgments, or reliability targets. We refer to such systematic updating of probability or frequency distributions across the PRA model as form of parametric fitting.

Recall from (Section~\ref{sec:unified_pra_dag}) that we represent a PRA model as a PDAG. Let \(\boldsymbol{\theta}\) be the collection of parameters governing all relevant probabilities/frequencies in this PDAG. For an end-state \(S_j\), the model-based prediction under \(\boldsymbol{\theta}\) is
\[
P_{\mathcal{M}}\bigl(S_j \mid \boldsymbol{\theta}\bigr).
\]
If one also has observed or target frequencies \(\bigl\{p_{j}^{\mathrm{obs}}\bigr\}\), parametric fitting seeks to reconcile this information with the model’s predictions by updating \(\boldsymbol{\theta}\). In a Bayesian setting, one may specify a prior distribution over \(\boldsymbol{\theta}\) and update this prior to a posterior distribution via the likelihood of observed end-state frequencies or other system-level evidence. Alternatively, one may adopt an optimization-based approach: define a loss or cost function that measures the discrepancy between \(\{p_{j}^{\mathrm{obs}}\}\) and \(\{P_{\mathcal{M}}(S_j \mid \boldsymbol{\theta})\}\), then minimize this loss with respect to \(\boldsymbol{\theta}\). Both perspectives aim to systematically adjust the PRA model’s probabilistic parameters so that end-state frequencies (or other risk metrics) remain consistent with available data or requirements. 

In the next section, we show how parametric fitting over the PDAG can be setup as a constrained optimization problem.

\section{Parameter Fitting as Constrained Optimization}
\label{sec:opt_formalization}

Each node \(X_i\) in the PDAG has an associated parameter \(\theta_i\), gathered into a vector  
\[
\boldsymbol{\theta}
\;=\;
(\theta_1,\;\theta_2,\;\dots,\;\theta_n).
\]
For a set of end-states \(\{S_j\}_{j=1}^m\), the model’s predicted probability under \(\boldsymbol{\theta}\) is  
\[
p_{j}^{\mathrm{pred}}\bigl(\boldsymbol{\theta}\bigr)
\;=\;
P_{\mathcal{M}}\bigl(S_j \mid \boldsymbol{\theta}\bigr).
\]
Suppose observed or target frequencies \(\bigl\{p_{j}^{\mathrm{obs}}\bigr\}\) are given. A discrepancy measure  
\[
d\!\bigl(p_{j}^{\mathrm{obs}},\,p_{j}^{\mathrm{pred}}(\boldsymbol{\theta})\bigr)
\]
compares the model’s predictions to these values. One can also add a regularization term \(\Psi(\boldsymbol{\theta})\) to encode additional constraints such as engineering limits or prior information. Let \(\Omega\) denote the feasible set for \(\boldsymbol{\theta}\), enforcing domain-specific requirements (e.g., probability normalization). Parameter fitting then becomes the following constrained optimization problem:
\[
\min_{\boldsymbol{\theta} \,\in\, \Omega} 
\quad 
\sum_{j=1}^m
d\!\Bigl(
   p_{j}^{\mathrm{obs}},\,
   p_{j}^{\mathrm{pred}}(\boldsymbol{\theta})
\Bigr)
\;+\;
\Psi(\boldsymbol{\theta}).
\]
A solution \(\boldsymbol{\theta}^{*}\) in \(\Omega\) is sought that minimizes overall discrepancy while respecting any additional constraints. Gradient-based methods (when \(d\) is differentiable) or other solvers can be employed.

\input{parts/4_learning/1_param/1_demo}
\input{parts/4_learning/1_param/3_case_study}
\input{parts/4_learning/1_param/4_figs}



\input{parts/1_foundations/1_PRA/0_intro}
\input{parts/1_foundations/1_PRA/1_ET}
\input{parts/1_foundations/1_PRA/2_FT}


\usetikzlibrary {intersections}
\usetikzlibrary {graphs}
\begin{figure}[ht!]
\centering
\begin{tikzpicture}[domain=-4:12]
  % \draw [help lines] (-4,0) grid (12,12);
  \draw[->] (-4.0,0) -- (11,0) node[right] {$x$};
  \draw[->] (-4,0) -- (-4,12.0) node[above] {$f(x)$};
\begin{scope}[every node/.style={thick}]
    \node (ss_begin) at (-2.75,1.75) {$\text{S}_0$};

    \node (ss_end) at (9.5,11.0) {$\text{ES}_{0}$} ;
    % \node (ss_1) at (2.5,0.0) {$S_1$};
    % \node (ss_i) at (5.0,0.0) {$S_i$};
    % \node (ss_j) at (10.0,0.0) {$S_j$};
    % \node (ss_jp1) at (12.5,0.0) {$S_{j+1}$};

\end{scope}

\begin{scope}[>={Circle[black]},
              every edge/.style={draw=black,very thick}]
    % \path [->] (ss_0) edge (ss_1);
    % \path [->] (ss_1) edge (ss_i);
    % \path [->] (ss_i) edge (ss_j);
    % \path [->] (ss_j) edge (ss_jp1);
    % \path [->] (ss_jp1) edge (ss_f);
    % %\path [->] (B) edge node {$3$} (C);
    %\path [<->] (ss_begin) edge[out=25,in=220] node {} (ss_ie_i); 
    \path[<->,save path=\pathA,name path=A] (ss_begin) to [out=45,in=160, edge node={node [near end, above] {$\text{S}_0$}}](ss_end);

    \path[<->,save path=\pathB,name path=B] (1,0) to (1,11);

  \fill[name intersections={of=A and B}] (intersection-1) circle (3pt);

  \node[anchor=south east] (ss_ie_i) at (intersection-1) {$\text{IE}_i$};

  \draw[<->,black,very thick][use path=\pathA];
  %\draw[red] [use path=\pathB];
    %\path [<->] (ss_begin) edge[out=25,in=180] node[near end, above] {$\text{S}_0$} (ss_end); 
    %\path [<->] (ss_ie_i) edge[out=0,in=180] node[sloped] {} (ss_end); 
\end{scope}
\begin{scope}
  [grow'=right,
   level distance=50pt,
   every tree node/.append style={anchor=base west},
   execute at begin node=\strut,
   level 0/.style={sibling distance=0pt},  % initiating event
   level 1/.style={sibling distance=55pt}, %
   level 2/.style={sibling distance=45pt},
   level 3/.style={sibling distance=25pt, nodes={right=2}},
   edge from parent/.append style={
        draw,
        edge from parent path={
            (\tikzparentnode.east) -| ($(\tikzparentnode.east)!0.5!(\tikzchildnode.west)$) |- (\tikzchildnode.west)
        },
    },
    every node/.append style={anchor=center,font=\small,text centered},
    % every level 0 node/.append style={circle, font=\small\bfseries, draw, fill=blue!30, inner sep=0pt},
    every leaf node/.append style={nodes={right=2}},
   ]
   \graph
{
  "$I$"[at={(intersection-1)}, right=1.25] -> {
    b -> c
  };
};

  % \node at (intersection-1)[right=1.25]{$I$}
  %    child {node {$F_{1}^s$}
  %      child {node {$F_{2}^s$}
  %        child {node {$\text{ES}_{1}$}}
  %      }
  %      child {node {$F_{2}^f$}
  %        child {node {$\text{ES}_{2}$}}
  %        child {node {$\text{ES}_{3}$}}
  %      }
  %    }
  %    child {node  {$F_{1}^f$}
  %      child {node {$\text{ES}_{4}$}}

  \end{scope}   
\end{tikzpicture}
\end{figure}

\chapter{Towards Parameter Fitting}
\label{sec:parametric_learning_pra_model}

PRAs invariably involve uncertainty. When explicitly modeled, these uncertainties can be updated or inferred from evidence, engineering judgments, or reliability targets. We refer to such systematic updating of probability or frequency distributions across the PRA model as form of parametric fitting.

Recall from (Section~\ref{sec:unified_pra_dag}) that we represent a PRA model as a PDAG. Let \(\boldsymbol{\theta}\) be the collection of parameters governing all relevant probabilities/frequencies in this PDAG. For an end-state \(S_j\), the model-based prediction under \(\boldsymbol{\theta}\) is
\[
P_{\mathcal{M}}\bigl(S_j \mid \boldsymbol{\theta}\bigr).
\]
If one also has observed or target frequencies \(\bigl\{p_{j}^{\mathrm{obs}}\bigr\}\), parametric fitting seeks to reconcile this information with the model’s predictions by updating \(\boldsymbol{\theta}\). In a Bayesian setting, one may specify a prior distribution over \(\boldsymbol{\theta}\) and update this prior to a posterior distribution via the likelihood of observed end-state frequencies or other system-level evidence. Alternatively, one may adopt an optimization-based approach: define a loss or cost function that measures the discrepancy between \(\{p_{j}^{\mathrm{obs}}\}\) and \(\{P_{\mathcal{M}}(S_j \mid \boldsymbol{\theta})\}\), then minimize this loss with respect to \(\boldsymbol{\theta}\). Both perspectives aim to systematically adjust the PRA model’s probabilistic parameters so that end-state frequencies (or other risk metrics) remain consistent with available data or requirements. 

In the next section, we show how parametric fitting over the PDAG can be setup as a constrained optimization problem.

\section{Parameter Fitting as Constrained Optimization}
\label{sec:opt_formalization}

Each node \(X_i\) in the PDAG has an associated parameter \(\theta_i\), gathered into a vector  
\[
\boldsymbol{\theta}
\;=\;
(\theta_1,\;\theta_2,\;\dots,\;\theta_n).
\]
For a set of end-states \(\{S_j\}_{j=1}^m\), the model’s predicted probability under \(\boldsymbol{\theta}\) is  
\[
p_{j}^{\mathrm{pred}}\bigl(\boldsymbol{\theta}\bigr)
\;=\;
P_{\mathcal{M}}\bigl(S_j \mid \boldsymbol{\theta}\bigr).
\]
Suppose observed or target frequencies \(\bigl\{p_{j}^{\mathrm{obs}}\bigr\}\) are given. A discrepancy measure  
\[
d\!\bigl(p_{j}^{\mathrm{obs}},\,p_{j}^{\mathrm{pred}}(\boldsymbol{\theta})\bigr)
\]
compares the model’s predictions to these values. One can also add a regularization term \(\Psi(\boldsymbol{\theta})\) to encode additional constraints such as engineering limits or prior information. Let \(\Omega\) denote the feasible set for \(\boldsymbol{\theta}\), enforcing domain-specific requirements (e.g., probability normalization). Parameter fitting then becomes the following constrained optimization problem:
\[
\min_{\boldsymbol{\theta} \,\in\, \Omega} 
\quad 
\sum_{j=1}^m
d\!\Bigl(
   p_{j}^{\mathrm{obs}},\,
   p_{j}^{\mathrm{pred}}(\boldsymbol{\theta})
\Bigr)
\;+\;
\Psi(\boldsymbol{\theta}).
\]
A solution \(\boldsymbol{\theta}^{*}\) in \(\Omega\) is sought that minimizes overall discrepancy while respecting any additional constraints. Gradient-based methods (when \(d\) is differentiable) or other solvers can be employed.

\input{parts/4_learning/1_param/1_demo}
\input{parts/4_learning/1_param/3_case_study}
\input{parts/4_learning/1_param/4_figs}


\chapter{Towards Parameter Fitting}
\label{sec:parametric_learning_pra_model}

PRAs invariably involve uncertainty. When explicitly modeled, these uncertainties can be updated or inferred from evidence, engineering judgments, or reliability targets. We refer to such systematic updating of probability or frequency distributions across the PRA model as form of parametric fitting.

Recall from (Section~\ref{sec:unified_pra_dag}) that we represent a PRA model as a PDAG. Let \(\boldsymbol{\theta}\) be the collection of parameters governing all relevant probabilities/frequencies in this PDAG. For an end-state \(S_j\), the model-based prediction under \(\boldsymbol{\theta}\) is
\[
P_{\mathcal{M}}\bigl(S_j \mid \boldsymbol{\theta}\bigr).
\]
If one also has observed or target frequencies \(\bigl\{p_{j}^{\mathrm{obs}}\bigr\}\), parametric fitting seeks to reconcile this information with the model’s predictions by updating \(\boldsymbol{\theta}\). In a Bayesian setting, one may specify a prior distribution over \(\boldsymbol{\theta}\) and update this prior to a posterior distribution via the likelihood of observed end-state frequencies or other system-level evidence. Alternatively, one may adopt an optimization-based approach: define a loss or cost function that measures the discrepancy between \(\{p_{j}^{\mathrm{obs}}\}\) and \(\{P_{\mathcal{M}}(S_j \mid \boldsymbol{\theta})\}\), then minimize this loss with respect to \(\boldsymbol{\theta}\). Both perspectives aim to systematically adjust the PRA model’s probabilistic parameters so that end-state frequencies (or other risk metrics) remain consistent with available data or requirements. 

In the next section, we show how parametric fitting over the PDAG can be setup as a constrained optimization problem.

\section{Parameter Fitting as Constrained Optimization}
\label{sec:opt_formalization}

Each node \(X_i\) in the PDAG has an associated parameter \(\theta_i\), gathered into a vector  
\[
\boldsymbol{\theta}
\;=\;
(\theta_1,\;\theta_2,\;\dots,\;\theta_n).
\]
For a set of end-states \(\{S_j\}_{j=1}^m\), the model’s predicted probability under \(\boldsymbol{\theta}\) is  
\[
p_{j}^{\mathrm{pred}}\bigl(\boldsymbol{\theta}\bigr)
\;=\;
P_{\mathcal{M}}\bigl(S_j \mid \boldsymbol{\theta}\bigr).
\]
Suppose observed or target frequencies \(\bigl\{p_{j}^{\mathrm{obs}}\bigr\}\) are given. A discrepancy measure  
\[
d\!\bigl(p_{j}^{\mathrm{obs}},\,p_{j}^{\mathrm{pred}}(\boldsymbol{\theta})\bigr)
\]
compares the model’s predictions to these values. One can also add a regularization term \(\Psi(\boldsymbol{\theta})\) to encode additional constraints such as engineering limits or prior information. Let \(\Omega\) denote the feasible set for \(\boldsymbol{\theta}\), enforcing domain-specific requirements (e.g., probability normalization). Parameter fitting then becomes the following constrained optimization problem:
\[
\min_{\boldsymbol{\theta} \,\in\, \Omega} 
\quad 
\sum_{j=1}^m
d\!\Bigl(
   p_{j}^{\mathrm{obs}},\,
   p_{j}^{\mathrm{pred}}(\boldsymbol{\theta})
\Bigr)
\;+\;
\Psi(\boldsymbol{\theta}).
\]
A solution \(\boldsymbol{\theta}^{*}\) in \(\Omega\) is sought that minimizes overall discrepancy while respecting any additional constraints. Gradient-based methods (when \(d\) is differentiable) or other solvers can be employed.

\input{parts/4_learning/1_param/1_demo}
\input{parts/4_learning/1_param/3_case_study}
\input{parts/4_learning/1_param/4_figs}


% \begin{figure}
% \begin{tikzpicture}
%   \graph [nodes={align=center, inner sep=1pt}, grow right=1.5]
% {
%   a,
%   b,
%   c -> d -> {
%     e -> f -> g,
%     h -> i
%   } -> j,
%   k -> l
% }
% \end{tikzpicture}
% \caption{An illustrative “success tree,” showing how multiple mitigation paths from the initial condition \(S_0\) can lead to safe or acceptable outcomes.}
% \label{fig:success_tree_example}
% \end{figure}
% Match the style from fig:event_tree_example
% \tikzset{grow'=right,level distance=48pt}
% %\tikzset{execute at begin node=\strut}
% %\tikzset{every tree node/.style={anchor=base west}}
% \tikzset{
%     edge from parent/.append style={very thick},
%     edge from parent/.style={
%         draw,
%         edge from parent path={
%             (\tikzparentnode.east) -| ($(\tikzparentnode.east)!0.5!(\tikzchildnode.west)$) |- (\tikzchildnode.west)
%         },
%     },
%     every node/.style={circle, minimum width=0.2cm, draw, anchor=center,font=\small\bfseries, text centered},
%     %every level 0 node/.style={circle, font=\small\bfseries, draw, fill=blue!30, inner sep=0pt},
%     %every internal node/.style={font=\small, inner sep=4pt},
%     %every leaf node/.style={rectangle, draw, fill=blue!30, minimum width=2.5cm, text centered},
%     %frontier/.style={distance from root=400pt},
% }
% \Tree [.\(1\)
%     [.\(F_1^{\text{fail}}\)
%         [.\(X_3\) ]
%     ]
% ]
% \Tree [.\(I\)
%     [.\(F_1^{\text{succ}}\)
%         [.\(F_2^{\text{succ}}\)
%             [.\(X_1\) ]
%         ]
%         [.\(F_2^{\text{fail}}\)
%             [.\(X_2\) ]
%         ]
%     ]
%     [.\(F_1^{\text{fail}}\)
%         [.\(X_3\) ]
%     ]
% ]
% Example success tree
% \Tree [.\(S_0\)
%     [.\(Mitigation \#1\)
%         [.\(Mitigation \#2\)
%             [.\(\text{Full Success}\) ]
%         ]
%         [.\(Alternate\)
%             [.\(\text{Partial Success}\) ]
%         ]
%     ]
%     [.\(Backup\)
%         [.\(\text{Alternate Success}\) ]
%     ]
% ]

% \part{A Brute Force Approach}
% \large{\begin{hindi}
% नर हो, न निराश करो मन को \\
% कुछ काम करो, कुछ काम करो
% \end{hindi}}
\chapter{Model Representation}

\input{parts/2_bruteforce/1_representation/1_DAG}
\input{parts/2_bruteforce/1_representation/2_alt_forms}
\chapter{Building a Data-Parallel Monte-Carlo Probability Estimator}

\input{parts/2_bruteforce/2_estimator/1_overview}

\input{parts/2_bruteforce/2_estimator/2_prng}

\section{Preliminary Benchmarks}

\input{parts/2_bruteforce/3_benchmark/3_table_aralia_ft_dataset}

\input{parts/2_bruteforce/3_benchmark/3_setup}

\clearpage
\begin{landscape}
\begin{figure}[h]
    \centering
    \includegraphics[width=1.2\textwidth]{parts/2_bruteforce/3_benchmark/error_vs_prob_detailed.png}
    \caption{Mean Absolute Error – Exact (BDD) vs Approximate Methods}
    \label{fig:mae_vs_logp}
\end{figure}
\end{landscape}

\input{parts/2_bruteforce/3_benchmark/4_mae}

\part{Refinements}
% \begin{sanskrit}
% करत करत अभ्यास, जड़मति होत सुजान
% \end{sanskrit}

\chapter{Efficient K-of-N Evaluation without Expansion}
\chapter{Variance Reduction}
\section{Dealing with Rare Events using Importance Sampling}
\subsection{Interplay between ultra-rare and ultra-frequent events - how they affect convergence}
\section{Sampling Correlated Events}
% \chapter{Hardware Optimizations}

\input{parts/3_refinements/2_hw_opt/1_voter/_}

\part{Inverse Problems}

\chapter{Towards Parameter Fitting}
\label{sec:parametric_learning_pra_model}

PRAs invariably involve uncertainty. When explicitly modeled, these uncertainties can be updated or inferred from evidence, engineering judgments, or reliability targets. We refer to such systematic updating of probability or frequency distributions across the PRA model as form of parametric fitting.

Recall from (Section~\ref{sec:unified_pra_dag}) that we represent a PRA model as a PDAG. Let \(\boldsymbol{\theta}\) be the collection of parameters governing all relevant probabilities/frequencies in this PDAG. For an end-state \(S_j\), the model-based prediction under \(\boldsymbol{\theta}\) is
\[
P_{\mathcal{M}}\bigl(S_j \mid \boldsymbol{\theta}\bigr).
\]
If one also has observed or target frequencies \(\bigl\{p_{j}^{\mathrm{obs}}\bigr\}\), parametric fitting seeks to reconcile this information with the model’s predictions by updating \(\boldsymbol{\theta}\). In a Bayesian setting, one may specify a prior distribution over \(\boldsymbol{\theta}\) and update this prior to a posterior distribution via the likelihood of observed end-state frequencies or other system-level evidence. Alternatively, one may adopt an optimization-based approach: define a loss or cost function that measures the discrepancy between \(\{p_{j}^{\mathrm{obs}}\}\) and \(\{P_{\mathcal{M}}(S_j \mid \boldsymbol{\theta})\}\), then minimize this loss with respect to \(\boldsymbol{\theta}\). Both perspectives aim to systematically adjust the PRA model’s probabilistic parameters so that end-state frequencies (or other risk metrics) remain consistent with available data or requirements. 

In the next section, we show how parametric fitting over the PDAG can be setup as a constrained optimization problem.

\section{Parameter Fitting as Constrained Optimization}
\label{sec:opt_formalization}

Each node \(X_i\) in the PDAG has an associated parameter \(\theta_i\), gathered into a vector  
\[
\boldsymbol{\theta}
\;=\;
(\theta_1,\;\theta_2,\;\dots,\;\theta_n).
\]
For a set of end-states \(\{S_j\}_{j=1}^m\), the model’s predicted probability under \(\boldsymbol{\theta}\) is  
\[
p_{j}^{\mathrm{pred}}\bigl(\boldsymbol{\theta}\bigr)
\;=\;
P_{\mathcal{M}}\bigl(S_j \mid \boldsymbol{\theta}\bigr).
\]
Suppose observed or target frequencies \(\bigl\{p_{j}^{\mathrm{obs}}\bigr\}\) are given. A discrepancy measure  
\[
d\!\bigl(p_{j}^{\mathrm{obs}},\,p_{j}^{\mathrm{pred}}(\boldsymbol{\theta})\bigr)
\]
compares the model’s predictions to these values. One can also add a regularization term \(\Psi(\boldsymbol{\theta})\) to encode additional constraints such as engineering limits or prior information. Let \(\Omega\) denote the feasible set for \(\boldsymbol{\theta}\), enforcing domain-specific requirements (e.g., probability normalization). Parameter fitting then becomes the following constrained optimization problem:
\[
\min_{\boldsymbol{\theta} \,\in\, \Omega} 
\quad 
\sum_{j=1}^m
d\!\Bigl(
   p_{j}^{\mathrm{obs}},\,
   p_{j}^{\mathrm{pred}}(\boldsymbol{\theta})
\Bigr)
\;+\;
\Psi(\boldsymbol{\theta}).
\]
A solution \(\boldsymbol{\theta}^{*}\) in \(\Omega\) is sought that minimizes overall discrepancy while respecting any additional constraints. Gradient-based methods (when \(d\) is differentiable) or other solvers can be employed.

\input{parts/4_learning/1_param/1_demo}
\input{parts/4_learning/1_param/3_case_study}
\input{parts/4_learning/1_param/4_figs}

\chapter{Towards Parameter Fitting}
\label{sec:parametric_learning_pra_model}

PRAs invariably involve uncertainty. When explicitly modeled, these uncertainties can be updated or inferred from evidence, engineering judgments, or reliability targets. We refer to such systematic updating of probability or frequency distributions across the PRA model as form of parametric fitting.

Recall from (Section~\ref{sec:unified_pra_dag}) that we represent a PRA model as a PDAG. Let \(\boldsymbol{\theta}\) be the collection of parameters governing all relevant probabilities/frequencies in this PDAG. For an end-state \(S_j\), the model-based prediction under \(\boldsymbol{\theta}\) is
\[
P_{\mathcal{M}}\bigl(S_j \mid \boldsymbol{\theta}\bigr).
\]
If one also has observed or target frequencies \(\bigl\{p_{j}^{\mathrm{obs}}\bigr\}\), parametric fitting seeks to reconcile this information with the model’s predictions by updating \(\boldsymbol{\theta}\). In a Bayesian setting, one may specify a prior distribution over \(\boldsymbol{\theta}\) and update this prior to a posterior distribution via the likelihood of observed end-state frequencies or other system-level evidence. Alternatively, one may adopt an optimization-based approach: define a loss or cost function that measures the discrepancy between \(\{p_{j}^{\mathrm{obs}}\}\) and \(\{P_{\mathcal{M}}(S_j \mid \boldsymbol{\theta})\}\), then minimize this loss with respect to \(\boldsymbol{\theta}\). Both perspectives aim to systematically adjust the PRA model’s probabilistic parameters so that end-state frequencies (or other risk metrics) remain consistent with available data or requirements. 

In the next section, we show how parametric fitting over the PDAG can be setup as a constrained optimization problem.

\section{Parameter Fitting as Constrained Optimization}
\label{sec:opt_formalization}

Each node \(X_i\) in the PDAG has an associated parameter \(\theta_i\), gathered into a vector  
\[
\boldsymbol{\theta}
\;=\;
(\theta_1,\;\theta_2,\;\dots,\;\theta_n).
\]
For a set of end-states \(\{S_j\}_{j=1}^m\), the model’s predicted probability under \(\boldsymbol{\theta}\) is  
\[
p_{j}^{\mathrm{pred}}\bigl(\boldsymbol{\theta}\bigr)
\;=\;
P_{\mathcal{M}}\bigl(S_j \mid \boldsymbol{\theta}\bigr).
\]
Suppose observed or target frequencies \(\bigl\{p_{j}^{\mathrm{obs}}\bigr\}\) are given. A discrepancy measure  
\[
d\!\bigl(p_{j}^{\mathrm{obs}},\,p_{j}^{\mathrm{pred}}(\boldsymbol{\theta})\bigr)
\]
compares the model’s predictions to these values. One can also add a regularization term \(\Psi(\boldsymbol{\theta})\) to encode additional constraints such as engineering limits or prior information. Let \(\Omega\) denote the feasible set for \(\boldsymbol{\theta}\), enforcing domain-specific requirements (e.g., probability normalization). Parameter fitting then becomes the following constrained optimization problem:
\[
\min_{\boldsymbol{\theta} \,\in\, \Omega} 
\quad 
\sum_{j=1}^m
d\!\Bigl(
   p_{j}^{\mathrm{obs}},\,
   p_{j}^{\mathrm{pred}}(\boldsymbol{\theta})
\Bigr)
\;+\;
\Psi(\boldsymbol{\theta}).
\]
A solution \(\boldsymbol{\theta}^{*}\) in \(\Omega\) is sought that minimizes overall discrepancy while respecting any additional constraints. Gradient-based methods (when \(d\) is differentiable) or other solvers can be employed.

\input{parts/4_learning/1_param/1_demo}
\input{parts/4_learning/1_param/3_case_study}
\input{parts/4_learning/1_param/4_figs}


%%---------------------------------------------------------------------------%%
%%  Bibliography 
%% or use BibTeX
\bibliography{references}
\bibliographystyle{apalike}

%%---------------------------------------------------------------------------%%
% Appendices
%\ensureoddstart
\restoregeometry

\appendix


\chapter{Revised Aralia Benchmark Plots}

In this chapter, we plot the the Monte--Carlo convergence experiments on the \emph{Aralia} fault--tree data set (Section~\ref{subsec:aralia_dataset}).  The revised study targeted a relative margin of error of $0.1\%$, that is, $\varepsilon = 10^{-3}\,\hat{p}$, at a $99\,\%$ confidence level with a wall--clock time limit of 60~s per model. The following figures collate the updated convergence traces for all 43 fault trees.

\begin{itemize}
  \item the sample mean estimate (solid colored line),
  \item the empirical $90\,\%$ and $99\,\%$ confidence bands (shaded regions)
  \item where available, the reference ``oracle/true'' probability (black dashed).
\end{itemize}


\begin{landscape}
\foreach \i in {1,...,9}{%%
  \begin{figure}[p]
      \centering
      \includegraphics[width=1\textwidth]{figs/convergence/e001p99/conv_fig_0\i.png}
      \caption{Aralia Fault Tree \i}
      \label{fig:conv_fig_\i}
  \end{figure}
}

\foreach \i in {10,...,43}{%%
  \begin{figure}[p]
      \centering
      \includegraphics[width=1\textwidth]{figs/convergence/e001p99/conv_fig_\i.png}
      \caption{Aralia Fault Tree \i}
      \label{fig:conv_fig_\i}
  \end{figure}
}
\end{landscape}

%\newgeometry{margin=1in,lmargin=1.25in,footskip=\chapterfootskip, includehead, includefoot}

% Can remove or add

\restoregeometry

%%---------------------------------------------------------------------------%%

%%---------------------------------------------------------------------------%%
\backmatter
\end{document}