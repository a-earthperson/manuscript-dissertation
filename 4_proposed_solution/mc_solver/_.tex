\chapter{Building a Data-Parallel Monte Carlo Probability Estimator}
\label{chap:mc_solver}

To handle massively parallel Monte Carlo evaluations of large-scale Boolean functions, we have developed a feedforward-layer architecture that organizes computation in a topological graph. At the lowest level, each Boolean variable/basic event (e.g., a component failure) is associated with a random number generator to sample its truth assignment. We bit-pack these outcomes, storing multiple Monte Carlo samples in each machine word to maximize computational throughput and reduce memory footprint. Subsequent layers consist of logically higher gates or composite structures that receive the bit-packed results from previous layers and combine them in parallel using coalesced kernels. By traversing the computation graph topologically, dependencies between gates and events are naturally enforced, so kernels for each layer can run concurrently once all prerequisite layers finish, resulting in high kernel occupancy and predictable throughput.

Section~\ref{sec:kernel_execution_model} formalizes how these kernels map the logical sampling workload onto a three–dimensional ND–range, introducing a consistent coordinate system $(i_x,i_y,i_z)$ and a rounding scheme that guarantees complete coverage without redundancy.  The remainder of this chapter adopts that notation.

In practice, each layer is dispatched to an accelerator node using a data-parallel model ed using \acrshort{sycl}. The random number generation pipelines are counter-based, ensuring reproducibility and thread-safety even across millions or billions of samples. Gates that go beyond simple AND/OR logic--such as \acrshort{vot} operators--are handled by specialized routines that can exploit native popcount instructions for efficient threshold evaluations. As we progress upwards through the layered topology, each gate or sub-function writes out its bit-packed output, effectively acting as an input stream to the next layer.
Throughout the simulation, online tallying kernels aggregate how often each node or gate evaluates to True. These tallies can then be turned into estimates of probabilities and sensitivity metrics on the fly. This approach also makes adaptive sampling feasible: if specific gates appear to dominate variance or are tied to particularly rare events, additional sampling can be allocated to their layer to refine estimates.

% -----------------------------------------------------------------------------
\section{Minimal Knowledge–Compilation Preprocessing for Monte-Carlo Sampling}
\label{sec:mc_kc_preprocessing}

Before any kernels are built, the solver applies a \emph{single, very light} compile pass whose only purpose is to shave off two gate types that would otherwise require special kernels:
\begin{itemize}
  \item \textbf{NULL gates} – a gate whose output is logically a no-op is deleted and any fan-out rewired directly to its input buffer.
  \item \textbf{NOT gates} – instead of scheduling a one-input gate kernel, we tag the affected buffer with an \emph{inversion flag}.  Every subsequent kernel simply toggles the word with a bitwise~\texttt{\char`~} on read.
\end{itemize}

\subsection{Why such little knowledge–compilation?}  Classical KC pipelines (Section ~\ref{sec:unified_pra_dag}) strive for determinism or decomposability to enable \emph{exact} inference. Monte-Carlo evaluation requires no such restrictions. No other syntactic normalization is attempted: negations may appear at arbitrary depth, XOR or $k/n$ gates remain untouched, and literals are free to occur with both polarities in different contexts.  This choice maximizes semantic expressivity and succinctness – flattening or pushing down is possible by using a higher compilation flag, but not strictly necessary.


% -----------------------------------------------------------------------------



\section{Layered Topological Organization}
\label{sec:layered_dag_traversal}

Recall that a \acrshort{pdag} \(\mathcal{G} = (\mathcal{V}, \mathcal{E})\) contains no cycles, so there is at least one valid \emph{topological ordering} of its nodes.  A topological ordering assigns each node a numerical \emph{layer index} such that all edges point from a lower-numbered layer to a higher-numbered layer. If a node \(v\) consumes the outputs of nodes \(\{u_1,\dots,u_k\}\), then we require
\[
\text{layer}(u_i) \;<\; \text{layer}(v)
\quad
\text{for each }i\in\{1,\dots,k\}.
\]
In other words, node \(v\) can appear only after all of its inputs in a linear or layered listing.

The essential steps to build and traverse these layers are:

\begin{enumerate}
    \item \emph{Compute Depths via Recursive Analysis:}  
      Each node's depth is found by inspecting its children (or inputs).  If a node is a leaf (e.g., a \texttt{Variable} or \texttt{Constant} that does not depend on any other node), its depth is 0.  Otherwise, its depth is one larger than the maximum depth among its children.  

    \item \emph{Group Nodes by Layer:}  
      Once each node's depth is computed, nodes of equal depth form a single \emph{layer}. Thus, all nodes with depth \(0\) are in the first layer, those with depth \(1\) in the second layer, and so on.  

    \item \emph{Sort Nodes within Each Layer:}  
      Within each layer, enforce an additional consistent ordering: (i)~variables appear before gates, (ii)~gates of different types can be grouped to facilitate specialized processing.  This step is not strictly required for correctness, but it can streamline subsequent stages such as kernel generation or partial evaluations.

    \item \emph{Traverse Layer by Layer:}  
      A final pass iterates over each layer in ascending order.  Because all inputs of any node in layer \(d\) lie in layers \(< d\), the evaluation (or "kernel build") for layer \(d\) can proceed after the entire set of layers \(0,\dots,d-1\) is processed.
\end{enumerate}

This structure ensures a sound evaluation of the \acrshort{pdag}: no gate or variable is computed until after all of its inputs are finalized.

\subsection{Depth Computation and Node Collection}

\begin{enumerate}
    \item \textbf{Clear Previous State.}  
      Any existing "visit" markers or stored depths in the \acrshort{pdag}-based data structures are reset to default values (e.g., zero or -1).
      
    \item \textbf{Depth Assignment by Recursion.}  
      A \texttt{compute\_depth} subroutine inspects each node:
      \begin{enumerate}
        \item If the node is a \texttt{Variable} or \texttt{Constant}, it is a leaf in the \acrshort{pdag}, so depth \(=0\).  
        \item If the node is a \texttt{Gate} with multiple inputs, the procedure first recursively computes the depths of its inputs. It then sets its own depth as 
        \[
          \text{depth}(\texttt{gate})
          \;=\;
          1 \;+\;\max\limits_{\ell \in \text{inputs of gate}} \Bigl[\text{depth}(\ell)\Bigr].
        \]
      \end{enumerate}
    \item \textbf{Order Assignment.}  
      Each node stores the newly computed depth in an internal field. This numeric value anchors the node to a layer. A consistent pass over the entire graph ensures correctness for all nodes.
\end{enumerate}

After depths are assigned, gather all nodes, walking the \acrshort{pdag} from its root, recording each discovered node and adding it to a global list.

\subsection{Layer Grouping and Local Sorting}

Begin by creating:
\begin{itemize}
\item A global list of all nodes, each with a valid depth,  
\item A mapping from node indices to node pointers,  
\end{itemize}
Then, sort the global list by ascending depth.  Let \(\text{order}(n)\) be the depth of node \(n\).  Then
\[
\text{order}(n_1)\;\le\;\text{order}(n_2)\;\le\;\dots\,\le\;\text{order}(n_{|\mathcal{V}|}).
\]
Finally, partition this list into contiguous \emph{layers}: if the deepest node has a depth \(\delta_{\max}\), then create sub-lists:
\[
\{\text{nodes s.t. depth}=0\},
\quad
\{\text{nodes s.t. depth}=1\},
\quad
\dots,
\quad
\{\text{nodes s.t. depth}=\delta_{\max}\}.
\]
Within each layer, sort nodes to ensure that \texttt{Variable} nodes precede \texttt{Gate} nodes, and \texttt{Gate} nodes may be further sorted by \texttt{Connective} type (e.g., \texttt{AND}, \texttt{OR}, \texttt{VOT}, etc.).

\subsection{Layer-by-Layer Kernel Construction}

Apply the layer decomposition to drive \emph{kernel building} and \emph{evaluation}:

\begin{enumerate}
    \item \textbf{Iterate over each layer in ascending depth}.  Because every node's dependencies lie in a strictly lower layer, one is guaranteed that those dependencies have already been assigned memory buffers, partial results, or other necessary resources.
    \item \textbf{Partition the layer nodes into subsets by node type}.  Concretely, \texttt{Variable} nodes are batched together for \emph{basic-event sampling} kernels, while \texttt{Gate} nodes are transferred into \emph{gate-evaluation} kernels.  
    \item \textbf{Generate device kernels}.  For \texttt{Variable} nodes, create Monte Carlo sampling kernels. For \texttt{Gate} nodes, it constructs logical or bitwise operations that merge or transform the sampled states of the inputs.  
\end{enumerate}

Once kernels for a given layer finish, move on to the next layer. Because of the topological guarantee, no node in layer \(d\) references memory or intermediate states from layer \(d\!+\!1\) or later, preventing cyclical references and ensuring correctness.

% -----------------------------------------------------------------------------
%  Extended algorithmic description of kernel generation and execution
% -----------------------------------------------------------------------------

\subsection{Basic--Event Sampling Kernels}
\label{subsec:be_kernel}

Each \texttt{Variable} node in layer~$d$ represents an i.i.d.~Bernoulli trial
with success probability~$p\_v\in[0,1]$.  The evaluation of all variables in
a layer is consolidated into 
\emph{one} data--parallel kernel that generates a contiguous block of
bit--packed outcomes:
\begin{enumerate}
  \item \textbf{Parameter staging.}  For every variable~$v$ the solver stores the
        pair $(\text{idx}(v),\,p\_v)$ in host memory, where
        $\text{idx}(v)$ is the global node index.  The list is stable across
        Monte--Carlo iterations and is therefore transferred to device memory
        only once.
  \item \textbf{Contiguous layout.}  A device--side array of $N_{\!v}$
        records, $N_{\!v}$ being the number of variables in the layer, is
        allocated so that the probability field, the bit--packed result buffer
        pointer, and any auxiliary counters are stored in
        \emph{structure--of--arrays} (SoA) form.  The resulting stride--free
        access pattern maximizes global--memory throughput.
  \item \textbf{Kernel configuration.}  Let $T$ denote the total number of
        Bernoulli draws requested by the host run--time (cf.
        Sec.~\ref{sec:bitpack-prob-sampling}).  The global ND--range is chosen
        as $\bigl(\lceil N_{\!v}\rceil,\,B,\,P\bigr)$, mapping each work--item
        to a unique triple $(v,\,b,\,p)$ of variable~$v$, batch index~$b$, and
        bit--pack index~$p$.  The local work--group shape is computed
        adaptively to saturate the target device while respecting hardware
        limits on registers and shared memory.
  \item \textbf{Execution.}  Every work--item initializes a counter--based
        generator (see the Philox discussion in
        Sec.~\ref{sec:bitpack-prob-sampling}), converts the pseudo--random words
        into $\omega$ Bernoulli outcomes via the integer--threshold technique,
        and writes the resulting $w$--bit word to the pre--allocated buffer.
        No inter--item synchronization is required beyond the implicit barrier
        at kernel completion.
\end{enumerate}
The overall cost is $\mathcal{O}(T\,N_{\!v}/\omega)$ arithmetic operations and
$\Theta(T\,N_{\!v}/\omega)$ global writes, making the routine
memory--bandwidth bound only for extremely small~$P$.

\subsection{Gate--Evaluation Kernels}
\label{subsec:gate_kernel}

Gate nodes are logically heterogeneous: AND, OR, XOR, NOT, NAND, NOR, XNOR, and
at--least--$k$ (\texttt{VOT}) gates all feature distinct Boolean semantics yet
share the same interface of reading one or more bit--packed input buffers and
writing a bit--packed output.  To avoid divergent control flow, the solver
instantiates \emph{one specialized kernel per connective type} present in the
current layer.

Consider a set $\mathcal{G}_{\mathrm{type}}$ containing all gates of a single
connective.  Their evaluation proceeds as follows:
\begin{enumerate}
  \item \textbf{Input resolution.}  For every gate $g\in\mathcal{G}$ the lists
        of positive inputs $\mathcal{I}^+(g)$ and negated inputs
        $\mathcal{I}^-(g)$ are resolved to concrete device pointers.  Positive
        and negative buffers are concatenated so that a simple offset marks the
        first negated operand.  The construction is embarrassingly parallel on
        the host and involves no device work.
  \item \textbf{Contiguous block construction.}  Buffers and gate metadata are
        packed into an SoA structure that is tile--aligned for
        coalesced reads.  For at--least--$k$ gates the threshold~$k$ is stored
        alongside the pointer list.
  \item \textbf{Kernel launch geometry.}  Let $N_{\!g}$ be the number of gates
        of the selected type.  An ND--range of
        $\bigl(\lceil N_{\!g}\rceil,\,B,\,P\bigr)$ is created, identical in
        shape to the basic--event kernel so that subsequent layers can reuse
        the same scheduling heuristics.  Within each work--item, Boolean logic
        is applied on a per--bitpack basis without branching:
        \begin{itemize}
          \item \textsc{And}, \textsc{Nand}:  multiple \texttt{\&} reductions
                plus an optional complement.
          \item \textsc{Or}, \textsc{Nor}:   multiple \texttt{|} reductions
                plus an optional complement.
          \item \textsc{Xor}, \textsc{Xnor}: accumulated parity via \texttt{\^{}} operations.
          \item \textsc{Null}, \textsc{Not}: trivial one--input, output, with complement.
          \item \textsc{At--least--$k$}: population counting of the aggregated
                bit--wise sum followed by a threshold comparison implemented
                through native \texttt{popcount} instructions.
        \end{itemize}
  \item \textbf{Dependency guarantees.}  Because all input buffers originate in
        earlier layers, the run--time enforces an event dependency on every
        producing kernel, ensuring visibility of the complete inputs before
        gate evaluation begins.
\end{enumerate}
The bit--parallel operations ensure that the arithmetic intensity is high; the
critical path is dominated by a handful of integer masks and, for
at--least--$k$ gates, one integer addition plus a comparison per input.

\subsection{Dependency--Aware Kernel Scheduling}
\label{subsec:scheduling}

Kernels are submitted to the device queue in strict layer order, yet the
scheduler exploits two orthogonal forms of parallelism:
\begin{enumerate}
  \item \emph{Intra--layer concurrency} --- basic--event sampling and the
        multiple gate kernels of the same layer depend exclusively on the
        previous layer, \emph{not} on one another.  They are therefore eligible
        for concurrent execution subject to device resources.
  \item \emph{Iterative sampling} --- the bit--packed sample space is sliced
        into $T\_\text{iter}$ iterations decided by the sample shaper.
        Kernels capturing the same node repeat across iterations and are
        expressed with an explicit iteration counter, enabling the run--time to
        re--use the same compiled binary while varying the random counter seed
        and output offsets.
\end{enumerate}
Dependencies are represented as light--weight events; the host never performs
explicit synchronization inside a layer but relies on the queue to enforce the
partial order.

\subsection{Work--Group Optimization Heuristics}
\label{subsec:wg_optim}

Let $G$ denote the global item count of the kernel at hand and $L\_\max$ the
maximum local size supported by the device along each axis.  The solver
selects a local range $(l_x,l_y,l_z)$ according to
\begin{align*}
  l_x &= \min\bigl(\text{pow2\_ceil}(G),\,L\_\max\bigr),\\
  l_y &= \min\bigl(B,\,L\_\max/l_x\bigr),\\
  l_z &= \min\bigl(P,\,L\_\max/(l_x l_y)\bigr),
\end{align*}
which heuristically balances occupancy with register pressure while retaining a
uniform work--item distribution.  The shape is re--evaluated independently for
basic events and gate kernels because $G$ differs across those two categories.

\subsection{Complexity and Scalability}
\label{subsec:complexity}

Assume $|\mathcal{V}|$ variables and $|\mathcal{G}|$ gates in the graph, with
layer depths bounded by $D$.  Let $S=T\,B\,P\,\omega$ be the total number of
Bernoulli trials.
\begin{itemize}
  \item \textbf{Kernel build time.}  All host--side preprocessing runs in
        $\mathcal{O}(|\mathcal{V}|+|\mathcal{G}|)$ memory operations; no search
        structure deeper than a hash map is required.
  \item \textbf{Device execution time.}  Each basic--event kernel performs
        $S$ integer comparisons.  Each gate kernel evaluates $S$ Boolean
        operations whose count is proportional to the fan--in of the gate.
        Hence the total arithmetic complexity is
        $\mathcal{O}\bigl(S\,(1+\overline{\deg})\bigr)$, where
        $\overline{\deg}$ is the average gate fan--in.
  \item \textbf{Parallel scalability.}  Both kernel categories exhibit linear
        speed--up with the number of compute units until either (i)~the global
        launch size no longer saturates the device or (ii)~memory bandwidth
        limits are reached.  Because all kernels are fully independent across
        the $B$ and $P$ dimensions, they scale particularly well on
        multi--tile accelerators.
\end{itemize}

The design therefore provides a clear separation of concerns: depth--first
analysis establishes the dependency structure; kernel generation translates
that structure into homogeneous, vectorizable work; and a light--weight event
system schedules the resulting kernels with minimal host intervention.

% -----------------------------------------------------------------------------
\section{Kernel-Level Execution Model}
\label{sec:kernel_execution_model}

\subsection{Coordinate System and Notation}

Let $(G_x,G_y,G_z)\in\mathbb{N}^3$ denote the \emph{global} range supplied to the device and $(L_x,L_y,L_z)$ the \emph{local} (work\,–\,group) range.  We further define
\[
  W_d \;=\; \frac{G_d}{L_d},\qquad d\in\{x,y,z\},
  \quad\text{and}\quad
  W \;=\; W_x W_y W_z ,
\]
where $W_d$ counts work-groups along axis~$d$ and $W$ is the total number of work-groups.  Every work-item within a group is identified by its local id $\ell=(\ell_x,\ell_y,\ell_z)$ with $0\le \ell_d<L_d$.

Unless stated otherwise the following global symbols are used throughout the chapter
\begin{center}
\begin{tabular}{ll}
$V$  & \# basic events (variables)\\
$G$  & \# standard logic gates\\
$A$  & \# at-least-$k$ gates\\
$T$  & Monte-Carlo iterations\\
$B$  & batches per iteration\\
$P$  & bit-packs per batch\\
$\omega$ & bits per pack $=8\cdot\mathrm{sizeof}(\texttt{bitpack\_t})$\\
$N$  & trials per iteration $=B\,P\,\omega$
\end{tabular}
\end{center}

\subsection{Generic Rounding Scheme}

All kernels adopt the \emph{nearest-multiple} rule
\[
   G_d \;=\;
   \Bigl\lceil \frac{Q_d}{L_d} \Bigr\rceil L_d ,\qquad
   Q_d \in \{V,G+A,1\}\times\{B\}\times\{P\},
\]
where $Q_d$ is the problem-specific lower bound listed in Table~\ref{tab:kernel_dimensions}.  This rule guarantees that every logical task is scheduled while respecting the SYCL constraint $G_d\equiv 0 \; (\mathrm{mod}\,L_d)$.

\subsection{Kernel-Specific Mappings}

Each kernel instantiates a surjective mapping
\[
   \Phi : \{0,\dots,G_x-1\}\times\{0,\dots,G_y-1\}\times\{0,\dots,G_z-1\}
          \;\twoheadrightarrow\; \mathcal{S},
\]
where $\mathcal{S}$ is the set of logical sub-tasks it must solve.  We list the mappings succinctly:

\begin{itemize}
  \item \textbf{Basic-event sampling} ($\#\mathcal{S}=VBP$):
        $\,\Phi_{\mathrm{BE}}(i_x,i_y,i_z)=(v=i_x,\;b=i_y,\;p=i_z)$.

  \item \textbf{Standard gate evaluation} ($\#\mathcal{S}=GBP$):
        $\,\Phi_{\mathrm{G}}(i_x,i_y,i_z)=(g=i_x,\;b=i_y,\;p=i_z)$.

  \item \textbf{At-least-$k$ gate evaluation}
        ($\#\mathcal{S}=ABP\omega$):
        $\,i_z = p\,\omega + \lambda$ with $\lambda\in\{0,\dots,\omega-1\}$.  The pair $(b,p)$ indexes the bit-pack, while $\lambda$ singles out a \emph{bit lane}.  One work-group therefore owns a unique triplet $(a,b,p)$ and folds the $\omega$ lanes with a group reduction.

  \item \textbf{Tally accumulation} ($\#\mathcal{S}=VBP$):
        identical to $\Phi_{\mathrm{BE}}$ but with $L_x=1$ such that each group covers exactly one tally node.
\end{itemize}

\subsection{Trial Coverage Guarantee}

Let $\Xi$ be the set of Bernoulli trials processed by a kernel in one iteration.  By construction
\[
   |\Xi|
   \;=\;
   \underbrace{B P \omega}_{\text{trials/ node}}
   \times
   \begin{cases}
     V, & \text{basic-event},\\[2pt]
     G, & \text{standard gate},\\[2pt]
     A, & \text{at-least gate},\\[2pt]
     V, & \text{tally}.
   \end{cases}
\]
Because $\omega$ divides $G_z$ in every case, each trial is owned by exactly one work-item and is executed precisely once.

\begin{table}[t]
  \centering
  \caption{Minimum global dimensions $Q_d$ before round-up.}
  \label{tab:kernel_dimensions}
  \begin{tabular}{lccc}
    \toprule
    Kernel               & $Q_x$              & $Q_y$ & $Q_z$\\
    \midrule
    Basic-event          & $V$                & $B$   & $P$\\
    Standard gate        & $G$                & $B$   & $P$\\
    At-least-$k$ gate    & $A$                & $B$   & $P\omega$\\
    Tally                & $V$                & $B$   & $P$\\
    \bottomrule
  \end{tabular}
\end{table}

\subsection{Work-Group Invariants}

Let $\Gamma$ be a work-group.  For every kernel the following invariant holds inside $\Gamma$:
\[
   \bigl[(\ell_x,\ell_y,\ell_z)\in\Gamma\bigr]
   \;\Longrightarrow\;
   \text{all work-items share the complete set of inputs required to produce one output literal}.
\]
Consequently intra-group communication (reductions, barriers) never crosses logical boundaries, enabling lock-free execution except for the single atomic update in the tally kernel.

\subsection{Complexity per Work-Group}

With $L=L_xL_yL_z$ the number of instructions executed by a group is
\[
  C_{\Gamma}
  \;=\;
  \begin{cases}
    \Theta(1), & \text{basic-event (bit-packing)},\\
    \Theta(\deg g), & \text{gate of fan-in }\deg g,\\
    \Theta(\deg a + \log \omega), & \text{at-least-}k,\\
    \Theta(L + \log L), & \text{tally popcount + reduction},
  \end{cases}
\]
all independent of $T$ owing to the strict buffering between iterations.

% -----------------------------------------------------------------------------
\chapter{Bitpacked Basic--Event Sampling Kernels}
\label{ch:prng-kernels}

Monte Carlo simulations, probability evaluations, and other sampling-based procedures benefit greatly from efficient, high-quality \acrfull{rng}s. A large class of modern \acrshort{rng}s are known as \textit{counter-based \acrfull{prng}s}, because they use integer counters (e.g., 32-bit or 64-bit) along with a stateless transformation to produce random outputs. The \emph{Philox} family of counter-based \acrshort{prng}s is a well-known example, featuring fast generation, high period, and good statistical properties. In this section, we discuss the general principles of counter-based \acrshort{prng}s, explain how Philox fits into this paradigm, analyze its complexity, and present a concise pseudocode version of the \(\text{Philox }4\times32\text{-10}\) variant. Subsequently, we detail the bitpacking scheme used to reduce memory consumption when storing large numbers of Bernoulli samples.

A counter-based \acrshort{prng} maps a user-supplied \emph{counter} (plus, optionally, a \emph{key}) to a fixed-size block of random bits via a deterministic function. Formally, if 
\[
  \mathbf{x} \;=\; (x_1, x_2, \ldots, x_k)
\]
is a vector of one or more 32-bit or 64-bit counters, and 
\[
  \mathbf{k} \;=\; (k_1, k_2, \ldots, k_m)
\]
is a key vector, then a counter-based \acrshort{prng} defines a transformation 
\[
   \mathcal{F}(\mathbf{x}, \mathbf{k})
   \;=\;
   (\rho_1, \rho_2, \ldots, \rho_r),
\]
where each \(\rho_j\) is typically a 32-bit or 64-bit output. Different increments of the counter \(\mathbf{x}\) produce different pseudo-random outputs \(\rho_j\). The process is stateless in the sense that advancing the RNG amounts to incrementing the counter (e.g., \(\mathbf{x}\mapsto \mathbf{x} + 1\)).

Compared to recurrence-based \acrshort{rng}s such as linear congruential generators or the Mersenne Twister, counter-based methods offer more straightforward parallelization, reproducibility across multiple streams, and strong structural simplicity: no internal state must be updated or maintained. This is particularly valuable in distributed Monte Carlo simulations or \acrshort{gpu}-based sampling, where each thread or work-item can be assigned a different counter. Philox constructs its pseudo-random outputs by applying a small set of mixed arithmetic (multiplication/bitwise) rounds to an input \emph{counter} plus \emph{key}. In particular, \(\mathrm{Philox}\,4\times32\text{-10}\) (often shortened to "Philox-4x32-10”) works on four 32-bit integers at a time:
\[
  \mathbf{S} = (S_0, S_1, S_2, S_3),
  \qquad
  \mathbf{K} = (K_0, K_1).
\]
The four elements \(\{S_0, S_1, S_2, S_3\}\) collectively represent the counter, e.g., \((x_0, x_1, x_2, x_3)\). The two key elements \((K_0, K_1)\) are used to tweak the generator's sequence. A single invocation of Philox-4x32-10 transforms \(\mathbf{S}\) into four new 32-bit outputs after ten rounds of mixing. At each round, the algorithm:
\begin{enumerate}
    \item Multiplies two of the state words by fixed "magic constants” to create partial products.
    \item Takes the high and low 32-bit portions of those 64-bit products.
    \item Incorporates the round key to shuffle the words.
    \item Bumps the key by adding constant increments \((\mathrm{W32A} = 0x9E3779B9 \text{ and } \mathrm{W32B} = 0xBB67AE85)\).
\end{enumerate}
After ten rounds, the final \((S_0, S_1, S_2, S_3)\) is returned as the pseudo-random block. A new call to Philox increases the counter \(\mathbf{S}\) by one (e.g., \(S_3 \mapsto S_3 + 1\)) and re-enters the same function. The Philox-4x32-10 algorithm is designed so that each blocking call requires a \emph{constant number} of operations, independent of the size of any prior "state.” Specifically, each round involves:
\[
  \mathcal{O}(1)\;\text{ arithmetic operations},
\]
and there are \(\mathrm{R} = 10\) rounds. Thus, each Philox invocation is asymptotically constant time \(\mathcal{O}(\mathrm{R}) = \mathcal{O}(1)\). The total cost to generate 128 bits (4 words \(\times\) 32 bits) is therefore constant time per call.

\section{The 10-round Philox-4x32}
Our implementation follows the standard 10-round approach for generating one block of four 32-bit random words, also called Philox-4x32-10. Let \(M_{\mathrm{A}}=0xD2511F53\), \(M_{\mathrm{B}}=0xCD9E8D57\) be the multipliers, and let \((K_0, K_1)\) be the key which is updated each round by \(\mathrm{W32A}=0x9E3779B9\) and \(\mathrm{W32B}=0xBB67AE85\). The function \(\text{Hi}(\cdot)\) returns the high 32 bits of a 64-bit product, and \(\text{Lo}(\cdot)\) returns the low 32 bits. Because each call produces four 32-bit pseudo-random words, Philox-4x32-10 is particularly convenient for batched sampling. If only a single 32-bit word is needed, one can still call the function and discard the excess words; however, many applications consume all four outputs (e.g., to produce four floating-point variates).

\begin{algorithm}[ht]
  \caption{Philox-4x32-10}\label{alg:philox}
  \begin{algorithmic}[1]
    %------------------------------------------------------------
    \Require Four 32-bit counters $(S_0,S_1,S_2,S_3)$,
            key $(K_0,K_1)$
    \Ensure  Transformed counters $(S_0,S_1,S_2,S_3)$
    %------------------------------------------------------------
    \Statex
    %--------------------- Philox_Round -------------------------
    \Procedure{Philox\_Round}{$(S_0,S_1,S_2,S_3),\,(K_0,K_1)$}
      \State $P_0 \gets M_{\text{A}}\times S_0$ \Comment{64-bit product}
      \State $P_1 \gets M_{\text{B}}\times S_2$ \Comment{64-bit product}
      \State $T_0 \gets \mathrm{Hi}(P_1)\,\oplus\,S_1\,\oplus\,K_0$
      \State $T_1 \gets \mathrm{Lo}(P_1)$
      \State $T_2 \gets \mathrm{Hi}(P_0)\,\oplus\,S_3\,\oplus\,K_1$
      \State $T_3 \gets \mathrm{Lo}(P_0)$
      \State $K_0 \gets K_0 + \mathrm{W32A}$
      \State $K_1 \gets K_1 + \mathrm{W32B}$
      \State \Return $\bigl((T_0,T_1,T_2,T_3),\,(K_0,K_1)\bigr)$
    \EndProcedure
    %------------------- Philox4x32_10 --------------------------
    \Statex
    \Procedure{Philox4x32\_10}{$(S_0,S_1,S_2,S_3),\,(K_0,K_1)$}
      \For{$i \gets 1$ \textbf{to} 10}
        \State $\bigl(S_0,S_1,S_2,S_3),\,(K_0,K_1) \gets$
               \Call{Philox\_Round}{$(S_0,S_1,S_2,S_3),\,(K_0,K_1)$}
      \EndFor
      \State \Return $(S_0,S_1,S_2,S_3)$
    \EndProcedure
  \end{algorithmic}
\end{algorithm}

\section{Bitpacking for Probability Sampling}
\label{sec:bitpack-prob-sampling}

It takes exactly one bit to represent the outcome of a Bernoulli trial. When these outcomes are stored naively, each occupies an entire \(8\)-bit byte, so only a fraction \(\tfrac{1}{8}\) of the allocated space carries useful information. Packing indicators into the native machine word of width \(w\) therefore reduces memory consumption by up to a factor of eight. More precisely,
\[
  \text{Memory}_{\text{naive}} \;=\; N \times 8\,\text{bits},
  \qquad
  \text{Memory}_{\text{packed}} \;=\;\Bigl\lceil\tfrac{N}{w}\Bigr\rceil \times w\,\text{bits},
\]
where \(N\) is the number of samples. In typical 64-bit environments we choose \(w = 64\), but the derivation is architecture-agnostic.

\subsection{Integer-threshold sampling.}  Instead of converting each 32-bit random word \(r\) produced by Philox into a floating-point variate, we compare it directly to an \emph{integer threshold}.  Let the target probability be \(p \in [0,1]\).  Define the threshold
\[
   T \;=\; \bigl\lfloor p \times 2^{32} \bigr\rfloor,\qquad 0 \le T \le 2^{32}-1.
\]
Because Philox delivers uniformly distributed 32-bit integers over \(\{0,\dots,2^{32}-1\}\), the event \(r < T\) occurs with probability exactly \(p\) (up to the discretization of the 32-bit grid).  The comparison is therefore sufficient to draw a Bernoulli outcome while avoiding the cost of division and floating-point arithmetic.  Using integer thresholds also guarantees identical behavior across heterogeneous hardware, an essential property for reproducible high-performance simulations.

\subsection{Grouping four comparisons.}  Each invocation of \(\mathrm{Philox\text{-}4\times32\text{-}10}\) yields the four words \(r_0, r_1, r_2, r_3\).  We evaluate the predicate \(r_j < T\) for every index \(j \in \{0,1,2,3\}\) and collect the four resulting bits into a 4-bit block.  The block is inserted into a wider accumulator using bitwise shifts.  Repeating the procedure fills the entire \(w\)-bit container.  If we denote by
\[
   g \;=\; \frac{w}{4}
\]
 the number of 4-bit generations required to populate the container, the overall cost is exactly \(g\) calls to the comparison–pack routine, or \(g\) additional increments of the counter component \(S_3\) in Philox.  No state other than the counter is maintained.

\begin{algorithm}[H]
  \caption{Integer-threshold packing of four Bernoulli outcomes}
  \label{alg:bernoulli_bitpack_int}
  \begin{algorithmic}[1]
    %------------------------------------------------------------
    \Require Probability $p\in[0,1]$; 32-bit words $(r_0,r_1,r_2,r_3)$
    \Ensure 4-bit integer \texttt{bits} containing the four Bernoulli draws
    %------------------------------------------------------------
    \State $T \gets \lfloor p\,2^{32}\rfloor$ \Comment{pre-compute once}
    \State \texttt{bits} $\gets 0$
    \For{$j \gets 0$ \textbf{to} $3$}
      \If{$r_j < T$}
        \State $b_j \gets 1$
      \Else
        \State $b_j \gets 0$
      \EndIf
      \State \texttt{bits} $\gets$ \texttt{bits} $\mid\, (b_j \ll j)$
    \EndFor
    \State \Return \texttt{bits}
  \end{algorithmic}
\end{algorithm}

\subsection{Assembling a complete bitpack.}  Calling Algorithm~\ref{alg:bernoulli_bitpack_int} successively for $g$ generations yields a full \(w\)-bit word whose $k$-th bit encodes the outcome of the $k$-th Bernoulli trial, ordered least-significant first.  The procedure is branch-free except for the threshold comparison and operates in $\mathcal{O}(g)$ time.  Because $g$ is fixed by the word size, the complexity is effectively constant per stored word, and the memory savings are realized without sacrificing statistical quality or portability.

% -----------------------------------------------------------------------------
\section{Counter Assignment Across the ND-Range}
\label{subsec:counter_assignment}

Let $(i_x,i_y,i_z)\in[0,G_x)\times[0,G_y)\times[0,G_z)$ denote the global id of a work-item in the basic-event kernel and let $t\in\{0,\dots,T-1\}$ be the Monte-Carlo iteration index.  A collision-free family of Philox counters is obtained via
\[
   \mathbf{C}(i_x,i_y,i_z,t)
   \;=\;
   \bigl(i_{x}+1,\, i_{z}+1,\, i_{y}+1,\, (t+1)\ll 6\bigr),
\]
where the six vacated least-significant bits of the fourth word are reserved for intra-kernel increments.  The mapping is bijective onto
\[
   \{1,\dots,V\}\times\{1,\dots,P\}\times\{1,\dots,B\}\times\{1,\dots,2^{6}\},
\]
guaranteeing independent pseudo-random streams for every work-item without synchronization or shared state.
% ============================================================================
%  Gate Kernels for Bit-Packed Boolean Evaluation
% ============================================================================
%  This file is manuscript-dissertation/4_proposed_solution/mc_solver/gate_kernels.tex
%  It is included from the parent chapter via \input{}.
% ----------------------------------------------------------------------------
\chapter{Gate Kernels for Bit\hyp{}Packed Boolean Evaluation}
\label{chap:gate_kernels}

\section{Connective Taxonomy}
\label{sec:gate_taxonomy}

Let $\mathcal{G}$ be the set of Boolean gates obtained from the topological
analysis of Section~\ref{sec:layered_dag_traversal}.  Each gate
$g\in\mathcal{G}$ is represented by the triplet
\[
  g = \bigl(\,\text{type}(g),\; \mathcal{I}^{+}(g),\; \mathcal{I}^{-}(g)\bigr),
\]
where $\mathcal{I}^{+}(g)$ and $\mathcal{I}^{-}(g)$ denote its positive and
negated inputs.  We partition $\mathcal{G}$ into disjoint subsets
$\mathcal{G}_{\mathrm{type}}$ according to
\[
  \text{type}(g)\in\bigl\{\textsc{Null},\textsc{Not},\textsc{And},\textsc{Or},
                   \textsc{Xor},\textsc{Nand},\textsc{Nor},\textsc{Xnor},\textsc{Atleast}\bigr\}.
\]
The subsequent sections analyze one subset at a time so that device kernels
remain branch\hyp{}free and resource usage is homogeneous.

\section{Launch Geometry}
\label{sec:gate_launch_geometry}

For every connective type we schedule one kernel with global range
\[
  (G_x,G_y,G_z)=\bigl(\lceil N_g\rceil,\;B,\;P\bigr),\qquad
  N_g = |\mathcal{G}_{\mathrm{type}}|,
\]
rounded by the nearest\hyp{}multiple rule of
Section~\ref{sec:kernel_execution_model}.  A work\hyp{}item with global id
$(i_x,i_y,i_z)$ therefore processes the unique triplet
\((g,b,p)\in\mathcal{G}_{\mathrm{type}}\times\{0,\dots,B-1\}\times\{0,\dots,P-1\}\).
Local ranges $(L_x,L_y,L_z)$ are chosen by the heuristic of
Section~\ref{subsec:wg_optim} and refined in
Section~\ref{subsec:gate_opt}.

\section{Bit\hyp{}Parallel Reduction Schemes}
\label{sec:gate_bitparallel}

\subsection{Idempotent Connectives: AND/OR Families}
\label{subsec:idempotent_connectives}

For $\textsc{And}$, $\textsc{Or}$ and their complements the kernel performs a
word\hyp{}wise left fold over the input list.  The accumulator is initialized
as
\[
  R_0 = \begin{cases}
           \texttt{AllOnes}, & \textsc{And}/\textsc{Nand},\\[4pt]
           \texttt{Zero},    & \textsc{Or}/\textsc{Nor}.
         \end{cases}
\]
Positive inputs use $R\gets R\otimes v$ with $\otimes\in\{\&,|\}$; negated
inputs substitute $\lnot v$.

\begin{lemma}[Bit\hyp{}wise Idempotence]
For any word size $\omega$ and any assignment of the input bits,
\[
  R_{\mathrm{final}}
  = \bigotimes_{u\in\mathcal{I}^{+}(g)} u\;
    \bigotimes_{v\in\mathcal{I}^{-}(g)} \lnot v
\]
yields a correct bit\hyp{}packed representation of gate $g$.
\end{lemma}

\begin{proof}
Idempotence of $\land$ and $\lor$ ensures order\hyp{}independent accumulation.
Per\hyp{}bit complement commutes with both operators, preserving semantics.
\end{proof}

The instruction count is $C_{\text{idemp}}=(\deg g)\,\Theta(1)$, one mask
operation per input, independent of $\omega$.

\subsection{Parity Connectives: \textsc{Xor}/\textsc{Xnor}}
\label{subsec:parity_connectives}

The fold operator becomes $\oplus$.  Associativity allows work\hyp{}groups to
split the input list and apply warp\hyp{}level reductions, lowering register
pressure for large fan\hyp{}ins (Section~\ref{subsec:gate_opt}).
A final complement realizes $\textsc{Xnor}$.

\subsection{Threshold Connectives: At\hyp{}Least\,$k$}
\label{subsec:threshold_connectives}

Let $n=\deg g$ and $k\in\{0,\dots,n\}$.  Fix the word width
$\omega=8\,\mathrm{sizeof}(\texttt{bitpack\_t})$ and launch each work\hyp{}group
with $L_z=\omega$ so that lane $\lambda\in\{0,\dots,\omega-1\}$ owns one bit
position.
\begin{enumerate}
  \item \emph{Per\hyp{}lane counting}: initialise $c_\lambda\gets0$; stream
        through the inputs, incrementing $c_\lambda$ whenever the masked bit is
        set (positive input) or cleared (negated input).
  \item \emph{Threshold test}: $r_\lambda\gets[c_\lambda\ge k]$.
  \item \emph{Group reduction}: a lane\hyp{}wise OR assembles the word
        $R=\sum_{\lambda} r_\lambda 2^{\lambda}$.
\end{enumerate}

\begin{theorem}[Work\hyp{}Group Correctness]
With the above geometry each work\hyp{}group writes exactly one valid output
word per iteration.
\end{theorem}

\begin{proof}
Bijectivity of the mapping $(\text{group},\text{lane})\mapsto(p,\lambda)$
guarantees single\hyp{}writer semantics; Steps~1–3 implement the at\hyp{}least\,$k$
predicate bit\hyp{}wise.
\end{proof}

The per\hyp{}lane cost is $n$ conditional increments plus one comparison;
adding the $\log_2\omega$\hyp{}step OR tree yields
$C_{\text{thr}}=\Theta(n+\log\omega)$.

\section{Performance Models}
\label{sec:gate_performance_models}

For idempotent and parity families let $I=n$ and memory traffic $M=n$.  Using
$B_{\text{mem}}$ and $\lambda$ from
Sections~\ref{subsec:cuda_backend}--\ref{subsec:cpu_backend},
\[
  \text{IPS}\;\le\;\min\Bigl(\frac{B_{\text{mem}}}{M w},\;\frac{C\,\lambda f}{I}\Bigr),
\] where $w$ is word size in bytes.  Threshold gates replace
$I\gets n+\log\omega$.

\section{Work\hyp{}Group Optimization Heuristics}
\label{subsec:gate_opt}

Empirically, gates with $\deg g>64$ profit from lane\hyp{}parallel counting
whereas smaller fan\hyp{}ins prefer maximal $l_y,l_z$ to saturate memory
bandwidth:
\[
  (l_x,l_y,l_z)=
  \begin{cases}
    (1,\,B,\,P), & \deg g>64,\\[4pt]
    (1,\,\min(B,L_{\max}),\,\min(P,L_{\max}/B)), & \text{otherwise}.
  \end{cases}
\]

\section{Complexity}
\label{sec:gate_complexity}

\[
  C_{\Gamma}=\begin{cases}
    \Theta(n), & \text{idempotent/parity},\\[4pt]
    \Theta(n+\log\omega), & \text{at\hyp{}least\,$k$}.
  \end{cases}
\]
Aggregated over all gates the arithmetic cost is
$\mathcal{O}\bigl(G\,n_{\mathrm{avg}} B P\bigr)$.

% ============================================================================ 
\chapter{Tallying Layer Outputs}
\label{sec:tally_kernel}

At every Monte-Carlo iteration the simulator produces, for each logic node
\(v\in \mathcal{V}\), a bit-packed buffer encoding
\[
  \mathbf{Y}_v^{(t)}
  \;=\;
  \bigl(y_{v,1}^{(t)}, y_{v,2}^{(t)},\dots, y_{v,N}^{(t)}\bigr)
  \in\{0,1\}^N,
  \quad t = 1,\dots,T,
\]
where \(N\!=\!B\!\times\!P\!\times\!\omega\) is the number of Bernoulli trials
per Monte-Carlo \emph{iteration}:
\begin{itemize}
    \item \(B\) - number of \emph{batches},
    \item \(P\) - bit-packs per batch,
    \item \(\omega\!=\!8\cdot\mathrm{sizeof}(\text{bitpack\_t})\) - bits per pack.
\end{itemize}
Because the buffers are overwritten at the next iteration, a
separate \emph{tally layer} accumulates summary statistics that persist for the
entire simulation.  The present section formalizes that process and outlines
an implementation-agnostic, data-parallel algorithm that realizes it on modern
accelerators.

\section{Statistical Objectives}
\label{subsec:tally_objective}

For every node \(v\) we wish to estimate, after \(T\) Monte-Carlo iterations,

\[
  \widehat{p}_v
  \;=\;
  \frac{1}{T\,N}
  \sum_{t=1}^{T}\sum_{j=1}^{N} y_{v,j}^{(t)}
  \;=\;
  \frac{s_v}{T\,N},
  \qquad
  s_v \;=\; \text{total \# of one-bits observed for node \(v\)}.
\]

Under the usual independence assumptions, the sampling distribution of
\(\widehat{p}_v\) is asymptotically
\[
\mathcal{N}\!\bigl(p_v,\,
  \tfrac{p_v(1-p_v)}{T\,N}\bigr)
\]
Hence

\[
  \widehat{\sigma}_v
  \;=\;
  \sqrt{\frac{\widehat{p}_v\,(1-\widehat{p}_v)}{T\,N}}
\]

is an unbiased estimator of the standard error, giving the
\((1-\alpha)\)\,--\,level normal confidence interval

\[
  \bigl[
    \widehat{p}_v - z_{1-\alpha/2}\,\widehat{\sigma}_v,\;
    \widehat{p}_v + z_{1-\alpha/2}\,\widehat{\sigma}_v
  \bigr],
  \qquad
  z_{1-\alpha/2}\in\{1.96,\,2.58,\dots\}.
\]

The tally routine therefore needs to maintain only the scalar
\(s_v\) while the simulation is running; the derived statistics can be updated
in-place whenever a user requests intermediate results or at a fixed cadence.

\section{Parallel Accumulation Algorithm}

The accumulation kernel is invoked on a three-dimensional
\texttt{nd\_range}, chosen such that
\[
  \begin{aligned}
    \text{global}_x &\;\ge\; V,\\
    \text{global}_y &\;\ge\; B,\\
    \text{global}_z &\;\ge\; P.
  \end{aligned}
\]
Work-item \((i_x,i_y,i_z)\) is responsible for \emph{exactly one} bit-pack:
\[
  \text{node  } v=i_x,\quad
  \text{batch } b=i_y,\quad
  \text{pack  } p=i_z.
\]

\vspace{4pt}
\noindent
\textbf{Local workflow of a work-item}
\begin{enumerate}
    \item Load the \(p^{\text{th}}\) bit-pack of batch \(b\) from
          \texttt{buffer}.
    \item Compute \(c=\mathrm{popcount}(\text{bitpack})\).
    \item Reduce the \(c\)'s belonging to the same work-\emph{group} in
          shared memory (tree reduction or \texttt{reduce\_over\_group}).
    \item One designated leader performs
          \(\texttt{atomic\_add}(\texttt{num\_one\_bits},\,\text{group\_sum})\).
\end{enumerate}

The reduction ensures only one atomic operation per group, greatly reducing
contention when \(P\) is large.

We present platform-neutral pseudocode that encapsulates the above logic while remaining agnostic to the underlying API. After each Monte-Carlo iteration the host enqueues \textsc{TallyKernel} with a
fresh \texttt{iteration} counter.  When either (i)~a user requests
intermediate statistics or (ii)~a pre-set reporting interval is reached,
the host reads back \texttt{num\_one\_bits} and executes the purely
serial routine shown in Algorithm~\ref{alg:update_stats}.

\begin{algorithm}[H]
\caption{Post-processing of a single node's tally}
\label{alg:update_stats}
\begin{algorithmic}[1]
  \Require
    \(s\) - total one-bits,
    \(T\), \(B\), \(P\), \(\omega\) - run parameters
  \Ensure
    \(\widehat{p}\), \(\widehat{\sigma}\), two symmetric CIs
  \State $N\gets B\cdot P\cdot\omega$
  \State $\widehat{p}\gets s / (T\,N)$
  \State $\widehat{\sigma}\gets
          \sqrt{\widehat{p}(1-\widehat{p})/(T\,N)}$
  \For{\textbf{each} $z\in\{1.96,\,2.58\}$}
      \State $\text{CI}\gets
        \bigl[\max(0,\widehat{p}-z\widehat{\sigma}),
              \min(1,\widehat{p}+z\widehat{\sigma})\bigr]$
  \EndFor
\end{algorithmic}
\end{algorithm}

The above normal approximation is valid provided \(T\,N\widehat{p}\)
and \(T\,N(1-\widehat{p})\) both exceed roughly 10; otherwise an exact
Clopper-Pearson interval can be substituted with no change to the running
sum logic.

\section{Correctness and Complexity}

\textbf{Work-item cost.}
Each work-item performs one \(\mathrm{popcount}\) and
participates in an \(O(\log L)\) intra-group reduction
(\(L\!=\!\text{local\_range}\)), yielding an overall
\(O(\log L)\) instruction count.

\textbf{Global cost.}
The total number of work-items launched per iteration is
\(V\cdot B\cdot P\).  Because each bit-pack contains \(\omega\) Bernoulli
trials, the cost \emph{per trial} shrinks as \(\omega^{-1}\).

\textbf{Memory traffic.}
Every work-item reads exactly one machine word and no writes occur except
the single atomic addition per work-group.  Hence the algorithm is
memory-bandwidth bound only at extremely low arithmetic intensity
(\(P\approx 1\)).

\textbf{Linear scalability.}
All tally nodes are independent.  Increasing \(V\) therefore scales the total
runtime linearly until either (i)~the device saturates its occupancy or
(ii)~atomic contention becomes non-negligible; the group-level reduction
mitigates the latter.

The design therefore provides a clear separation of concerns: depth--first
analysis establishes the dependency structure; kernel generation translates
that structure into homogeneous, vectorizable work; and a light--weight event
system schedules the resulting kernels with minimal host intervention.

% -----------------------------------------------------------------------------
%  Extended implementation-oriented discussion (matches the realised kernel)
% -----------------------------------------------------------------------------

\section{Work--Group Geometry and Synchronization}
\label{subsec:tally_geometry}

The three--dimensional launch geometry \((v,\,b,\,p)\) outlined in
Sec.~\ref{subsec:tally_objective} is refined in the implementation to minimize
both occupancy loss and atomic contention.  A crucial design choice is to fix
\emph{local}~\(x=1\), thereby dedicating one work--group to exactly one tally
node~\(v\).  The remaining two dimensions then tile the \((b,p)\)--plane with a
rectangular block of size
\[
  \bigl(1,\,l_y,\,l_z\bigr),
  \qquad l_y\cdot l_z\le L_{\max},
\]
where \(L_{\max}\) is the device--specific upper bound on the total work--group
size.  Provided \(l_y\!\ge\!B\) and \(l_z\!\ge\!P\), only \emph{one} group is
dispatched per tally and per iteration, guaranteeing that the reduction of
Step~3 and the atomic addition of Step~4 in
Sec.~\ref{sec:tally_kernel} execute exactly once.  Should resource pressure
force \(l_y< B\) or \(l_z< P\), multiple groups are launched and the atomic
update is replicated; correctness is preserved by the commutativity of
addition, but the repeated work incurs a small overhead.  The occupancy model
therefore trades a moderate loss in parallelism for deterministic behavior and
reduced synchronization cost.

A relaxed memory order is sufficient for the atomic accumulator because the
kernel guarantees \emph{program order} between the intra--group reduction and
the atomic~\texttt{fetch\_add}.  No additional fences are required, and the
resulting implementation maps efficiently to both discrete and integrated
\acrshort{gpu}s.

\section{Incremental Update of Derived Statistics}
\label{subsec:tally_stats_refresh}

While Monte--Carlo sampling proceeds, applications often request intermediate
probability estimates~\(\widehat{p}_v^{(t)}\) before the total budget~\(T\) is
exhausted.  Recomputing \(\widehat{p}_v\) and
\(\widehat{\sigma}_v\) from scratch would require a host round--trip for every
sampled bit.  Instead, the tally layer maintains two scalars per node:
\(s_v\) (total one--bits) and \(n_v\) (total bits processed).  After each
completed iteration the host merely increments \(n_v\gets n_v + N\) and leaves
\(s_v\) to the device kernel.  Whenever a refresh is requested the statistics
are updated via
\[
  \widehat{p}_v\;=\;\frac{s_v}{n_v},
  \qquad
  \widehat{\sigma}_v\;=\;\sqrt{\frac{\widehat{p}_v(1-\widehat{p}_v)}{n_v}},
\]
which costs \(\mathcal{O}(V)\) host--side arithmetic and no device work.  In
practice the refresh cadence is set adaptively: frequent updates early in the
run aid variance monitoring, whereas late--stage updates can be spaced further
apart because the relative change in \(\widehat{p}_v\) diminishes as
\(n_v\to T\,N\).

\section{Convergence Diagnostics and Stopping Rules}
\label{subsec:tally_convergence}

Two families of diagnostics leverage the quantities already maintained by the
tally kernel:
\begin{enumerate}
  \item\textbf{Relative half--width criterion.}  Define the
        half--width of the \((1-\alpha)\)--level interval as
        \(h_v= z_{1-\alpha/2}\,\widehat{\sigma}_v\).  The run may be terminated
        for node~\(v\) once \(h_v/\widehat{p}_v\le \varepsilon\), where
        \(\varepsilon\) is a user--supplied tolerance.  Because both
        \(\widehat{\sigma}_v\) and \(\widehat{p}_v\) are inexpensive to update,
        the test incurs negligible overhead.
  \item\textbf{Sequential Wald test.}  When the goal is to decide whether
        \(p_v\) exceeds a safety threshold~\(p_0\), one may adopt the
        sequential probability ratio test with boundaries derived from
        \(s_v\) and \(n_v\).  The tally structure already provides the minimal
        sufficient statistics, so the host evaluates the Wald condition after
        every refresh with no additional device interaction.
\end{enumerate}
Because the diagnostics rely solely on \(s_v\) and \(n_v\), no modification to
the kernel is needed; all logic resides in a lightweight host callback.

\section{Implementation Cost Model}
\label{subsec:tally_cost_model}

Let \(C_{\mathrm{pc}}\) denote the latency of a hardware popcount and
\(C_{\mathrm{rd}}(l)\) the latency of a tree reduction over \(l\) work--items.
The wall--clock time per iteration is approximated by
\[
  T_{\text{iter}} \;\approx\;
  (C_{\mathrm{pc}} + C_{\mathrm{mem}})\,VBP +
  C_{\mathrm{rd}}(l_y l_z)\,\frac{VBP}{l_y l_z}
  + C_{\mathrm{atomic}}\,\frac{VBP}{l_y l_z},
\]
where \(C_{\mathrm{mem}}\) and \(C_{\mathrm{atomic}}\) are the per--word memory
and atomic latencies, respectively.  The model highlights two regimes:
\begin{itemize}
  \item\emph{Arithmetic bound}: when \(P\gg 1\) and the popcount throughput
        saturates the execution units, the first term dominates and scaling is
        limited by instruction bandwidth.
  \item\emph{Memory bound}: when \(P\approx 1\) the workload collapses to a
        single read per work--item; the kernel becomes memory bandwidth--
        limited as predicted in Sec.~\ref{subsec:tally_objective}.
\end{itemize}


\section{Numerical Robustness}
\label{subsec:tally_numerics}

All accumulators operate in integer arithmetic, thereby eliminating rounding
error in \(s_v\).  Derived quantities computed in double precision satisfy
\(\lvert\widehat{p}_v - s_v/n_v\rvert < 2^{-53}\), well below any practical
error criterion for reliability analysis.  Clamping the confidence interval
bounds to~\([0,1]\) prevents pathological estimates when either \(s_v=0\) or
\(s_v=n_v\) in early iterations.

% -----------------------------------------------------------------------------
\section{Relation to the Global Execution Model}
\label{subsec:tally_exec_relation}

The specialised geometry adopted in Section~\ref{subsec:tally_geometry} is a direct instantiation of the rules formalised in Section~\ref{sec:kernel_execution_model}.  Choosing $L_x=1$ enforces the work\,–\,group invariant whereby a group owns exactly one tally node while still satisfying
\[
   G_x \;=\; \Bigl\lceil \frac{V}{L_x} \Bigr\rceil L_x \;=\; V ,
\]
so no over-provisioning occurs along the $x$-axis.  The remaining dimensions follow the generic rounding scheme with $(Q_y,Q_z)=(B,P)$, thus preserving the one-to-one correspondence between work-items and bit-packs established in Section~\ref{sec:kernel_execution_model}.

% ============================================================================
%  Backend-Specific Execution Mapping and Scalability Analysis (revised)
% ============================================================================
%  This file is manuscript-dissertation/4_proposed_solution/mc_solver/backends.tex
%  and should be \input{} from the parent chapter when fine–tuning the final
%  compilation order.
% ----------------------------------------------------------------------------
\chapter{Backend–Specific Scalability Analysis}
\label{sec:backend_scaling}

In this section we instantiate the abstract execution model of
Section~\ref{sec:kernel_execution_model} on two concrete hardware backends that
span the current spectrum of commodity accelerators:
\emph{(i)} NVIDIA GPUs programmed through the CUDA tool-chain and
\emph{(ii)} shared-memory multicore CPUs equipped with wide SIMD units.
The discussion follows the roofline methodology~\cite{Williams2009Roofline}
wherever a bandwidth–or–compute bottleneck must be highlighted and retains the
global kernel symbols introduced previously.  Additional backend parameters
are summarized in Table~\ref{tab:backend_params}; architectural constants are
set in \emph{italic type}.

\begin{table}[t]
  \centering
  \caption{Backend parameters introduced in this section.  Architectural
           constants are shown in \emph{italic}.}
  \label{tab:backend_params}
  \begin{tabular}{ll}
    \toprule
    Symbol & Meaning\\
    \midrule
    $\mathit{C}$         & physical compute units (SMs on CUDA, cores on CPU)\\
    $\mathit{W_s}$       & SIMD-lane width ("warp" size on CUDA; 32 on recent GPUs)\\
    $T_{\max}$           & maximum resident work-items per work-group / block\\
    $B_{\max}$           & scheduler limit on concurrent blocks per compute unit\\
    $R_{\max}$           & registers available per compute unit\\
    $B_{\text{mem}}$     & attainable device memory bandwidth\\
    $f$                  & core clock frequency (Hz)\\
    \bottomrule
  \end{tabular}
\end{table}

Throughout we denote by $L=L_xL_yL_z$ the total number of work-items in a
SYCL work-group and by $W=W_xW_yW_z$ the total number of work-groups launched
by the kernel.

% ----------------------------------------------------------------------------
\section{CUDA GPU backend}
\label{subsec:cuda_backend}

\subsection{Thread-Block Mapping}
Each SYCL work-group is lowered to a CUDA \emph{thread block}.  The effective
block size used by the hardware is therefore
\[
  L_{\text{CUDA}}\;=\;\min(L,\,T_{\max}).
\]
The kernel grid retains the user-specified dimensions $(W_x,W_y,W_z)$ and thus
launches $W$ blocks in total.  A block of size $L_{\text{CUDA}}$ contains
$\lceil L_{\text{CUDA}}/W_s\rceil$ warps.

\subsection{Theoretical Occupancy}
A standard proxy for latency hiding on GPUs is the \emph{occupancy}
$\mathcal{O}$, i.e. the ratio between active and maximum resident warps per
SM.  With
\[
  W_{\text{act}} \;=\; \min\!\bigl(B_{\max},\,\bigl\lceil\tfrac{L_{\text{CUDA}}}{W_s}\bigr\rceil\bigr)\times
                     \bigl\lceil\tfrac{W}{C}\bigr\rceil
\]
active per-SM warps, the theoretical occupancy evaluates to
\[
  \mathcal{O}_{\text{th}} \;=\; \min\!\Bigl(1,\,\frac{W_{\text{act}}}{B_{\max}T_{\max}/W_s}\Bigr).
\]
Given that realistic Monte-Carlo workloads satisfy $W\gg C$ the rightmost
fraction approaches~1, and kernels are typically either register- or
shared-memory-limited rather than scheduler-limited.

\subsection{Register Footprint and Latency Hiding}
Let $R$ denote the per-thread register footprint measured by the compiler and
$W_{\text{reg}}=\lfloor R_{\max}/(RW_s)\rfloor$ the register-constrained warp
capacity per SM.  The empirical latency hiding factor on modern NVIDIA
hardware can be captured by
\[
  \lambda_{\text{CUDA}} = \min(W_{\text{reg}},\,W_s B_{\max}),
\]
which saturates instruction throughput once $\lambda_{\text{CUDA}}\gtrsim 4$.

\subsection{Throughput Model}
Let $I$ be the static instruction count per thread derived in
Sec.~\ref{sec:kernel_execution_model}.  Assuming that the kernel is
compute-bound the sustained instruction rate (IPS) is approximated by
\[
  \text{IPS}_{\text{CUDA}} \;\approx\; \frac{C\,\lambda_{\text{CUDA}}\,f}{I}.
\]
The expression is linear in both the number of compute units~$C$ and the
latency hiding factor~$\lambda_{\text{CUDA}}$ until the memory subsystem is
saturated; the break-even point is estimated in the roofline plot of
Fig.~\ref{fig:roofline_cuda} (omitted here for brevity).

% ----------------------------------------------------------------------------
\section{Shared-Memory Multicore CPU Backend}
\label{subsec:cpu_backend}

\subsection{Work-Group to Thread Mapping}
On CPUs a SYCL work-group is translated to an OpenMP \verb|parallel for|
\emph{team}.  The default team size equals $L$ but cannot exceed the
architectural limit $T_{\max}=W_s$.  The outermost loop distributes the $W$
work-groups over the available hardware threads, i.e. over $C$ physical cores
and their simultaneous multithreading (SMT) contexts.

\subsection{Vectorization Strategy}
The innermost kernel dimension (global~$z$) holds independent bit-pack
indices.  Mapping that dimension to SIMD lanes yields perfect utilization as
long as the kernel exposes at least $W_s$ independent bit-packs, which is
guaranteed for the Monte-Carlo sample sizes considered ($\omega\ge W_s$).

\subsection{Roofline Bound}
With $b$~bytes and $i$~floating-point instructions issued per trial the classical roofline
model bounds the attainable performance by
\[
  P_{\text{CPU}} \;\le\; \min\!\Bigl(\frac{B_{\text{mem}}}{b},\;\frac{C\,I_{\text{F}}}{i}\Bigr),
\]
where $I_{\text{F}}$ denotes the peak per-core fused-multiply-add (FMA) rate.
For the present kernels the operational intensity $i/b\approx 0.25\,$FMA/B
places almost all CPU runs in the bandwidth-bound regime unless the sample
count per node exceeds $10^{10}$, well beyond typical reliability studies.

\subsection{Strong-Scaling Perspective}
Holding the global problem size fixed while increasing the core count leads to
a speed-up characterized empirically by
\[
  S(C) \;=\; \frac{T_1}{T_C} \;\approx\; \frac{C}{1 + \alpha(C-1)},
\]
where the serial fraction $\alpha\le0.05$ was obtained on a 64-core Zen4
system.  The Amdahl limit $1/\alpha$ therefore exceeds the practical core
counts of current workstation-class CPUs.

\subsection{Practical Parameter Choices}
Extensive auto-tuning on a 64-core Zen4 host and an NVIDIA Ada GPU suggests
\begin{center}
  \setlength{\tabcolsep}{6pt}
  \begin{tabular}{lcc}
     \toprule
     Kernel class & GPU (CUDA) & CPU (OpenMP)\\
     \midrule
     \textbf{Tally} & $(L_x,L_y,L_z)=(1,\,W_s,\,1)$ & $(1,\,W_s,\,1)$\\
     \textbf{Gate}  & $(1,\,1,\,W_s)$ & $(1,\,1,\,W_s)$\\
     \bottomrule
  \end{tabular}
\end{center}
which aligns the innermost loop with the cache line size and the SIMD width on
both architectures.

% ============================================================================


\chapter{Towards Parameter Fitting}
\label{sec:parametric_learning_pra_model}

PRAs invariably involve uncertainty. When explicitly modeled, these uncertainties can be updated or inferred from evidence, engineering judgments, or reliability targets. We refer to such systematic updating of probability or frequency distributions across the PRA model as form of parametric fitting.

Recall from (Section~\ref{sec:unified_pra_dag}) that we represent a PRA model as a PDAG. Let \(\boldsymbol{\theta}\) be the collection of parameters governing all relevant probabilities/frequencies in this PDAG. For an end-state \(S_j\), the model-based prediction under \(\boldsymbol{\theta}\) is
\[
P_{\mathcal{M}}\bigl(S_j \mid \boldsymbol{\theta}\bigr).
\]
If one also has observed or target frequencies \(\bigl\{p_{j}^{\mathrm{obs}}\bigr\}\), parametric fitting seeks to reconcile this information with the model’s predictions by updating \(\boldsymbol{\theta}\). In a Bayesian setting, one may specify a prior distribution over \(\boldsymbol{\theta}\) and update this prior to a posterior distribution via the likelihood of observed end-state frequencies or other system-level evidence. Alternatively, one may adopt an optimization-based approach: define a loss or cost function that measures the discrepancy between \(\{p_{j}^{\mathrm{obs}}\}\) and \(\{P_{\mathcal{M}}(S_j \mid \boldsymbol{\theta})\}\), then minimize this loss with respect to \(\boldsymbol{\theta}\). Both perspectives aim to systematically adjust the PRA model’s probabilistic parameters so that end-state frequencies (or other risk metrics) remain consistent with available data or requirements. 

In the next section, we show how parametric fitting over the PDAG can be setup as a constrained optimization problem.

\section{Parameter Fitting as Constrained Optimization}
\label{sec:opt_formalization}

Each node \(X_i\) in the PDAG has an associated parameter \(\theta_i\), gathered into a vector  
\[
\boldsymbol{\theta}
\;=\;
(\theta_1,\;\theta_2,\;\dots,\;\theta_n).
\]
For a set of end-states \(\{S_j\}_{j=1}^m\), the model’s predicted probability under \(\boldsymbol{\theta}\) is  
\[
p_{j}^{\mathrm{pred}}\bigl(\boldsymbol{\theta}\bigr)
\;=\;
P_{\mathcal{M}}\bigl(S_j \mid \boldsymbol{\theta}\bigr).
\]
Suppose observed or target frequencies \(\bigl\{p_{j}^{\mathrm{obs}}\bigr\}\) are given. A discrepancy measure  
\[
d\!\bigl(p_{j}^{\mathrm{obs}},\,p_{j}^{\mathrm{pred}}(\boldsymbol{\theta})\bigr)
\]
compares the model’s predictions to these values. One can also add a regularization term \(\Psi(\boldsymbol{\theta})\) to encode additional constraints such as engineering limits or prior information. Let \(\Omega\) denote the feasible set for \(\boldsymbol{\theta}\), enforcing domain-specific requirements (e.g., probability normalization). Parameter fitting then becomes the following constrained optimization problem:
\[
\min_{\boldsymbol{\theta} \,\in\, \Omega} 
\quad 
\sum_{j=1}^m
d\!\Bigl(
   p_{j}^{\mathrm{obs}},\,
   p_{j}^{\mathrm{pred}}(\boldsymbol{\theta})
\Bigr)
\;+\;
\Psi(\boldsymbol{\theta}).
\]
A solution \(\boldsymbol{\theta}^{*}\) in \(\Omega\) is sought that minimizes overall discrepancy while respecting any additional constraints. Gradient-based methods (when \(d\) is differentiable) or other solvers can be employed.

\section{Case Study: EBR-II Liquid Metal Fire Scenario}
\label{sec:case_study_ebr2}
We apply the proposed optimization method to an event tree from the Experimental Breeder Reactor-II (EBR-II) Level I PRA \citep{chang_experimental_2018}. The potential initiating event is a leak in the piping loop of the reactor's shutdown cooler, which uses sodium-potassium (NaK) coolant. Air intrusion near NaK can cause fire hazards. The event tree, shown in Figure \ref{fig:ebr2_sdfr_et} enumerates whether (i)~the liquid-metal fire is detected in time (\(\text{LMFD}\)), (ii)~a reactor scram is successfully initiated (\(\text{RFIR}\)), (iii)~the fire is classified as severe or limited (\(\text{LLRF}\)), (iv)~a plant-level fire suppression system fails or succeeds (\(\text{SSSD}\)), and (v)~critical secondary systems remain operational (\(\text{SYSO}\)). These conditional events interact to form multiple end-states, labeled \(\text{SDFR-0}\) through \(\text{SDFR-8}\). Some end-states represent minimal impact (e.g., immediate fire detection and promptly executed scram), whereas others lead to more severe conditions (e.g., no detection and system failures yielding potential core damage).

\begin{figure}[h]
  \centering
\includegraphics[width=1.0\textwidth]{parts/4_learning/1_param/figs/event_tree_NaK_fire.png} 
    \caption{EBR-II Shutdown Cooler NaK Fire in Containment}
    \label{fig:ebr2_sdfr_et}
\end{figure}

\subsection{Event Tree Structure and Problem Setup}
Following the notation from Section~\ref{sec:event_tree_definition}, each end-state \(S_j\) arises from a particular path of success/failure outcomes across the conditional events. Let \(\{X_1,\dots,X_n\}\) be the events (e.g., \(\text{LMFD}, \text{RFIR}, \dots\)), and let \(y_{ji}\in\{0,1,\text{NaN}\}\) indicate whether \(X_i\) fails, succeeds, or is not applicable for path \(S_j\). The probability of end-state \(S_j\) is
\begin{equation}
\label{eq:ebr2_es_probability}
P(S_j) 
\;=\;
\prod_{i=1}^{n} P\bigl(y_{ji}\bigr),
\end{equation}
where 
\begin{equation}
P\bigl(y_{ji}\bigr)
\;=\;
\begin{cases}
P\bigl[X_i = 1\bigr], & \text{if } y_{ji}=1,\\
1 - P\bigl[X_i = 1\bigr], & \text{if } y_{ji}=0,\\
1, & \text{if } y_{ji}=\text{NaN}.
\end{cases}
\label{eq:ebr2_conditional}
\end{equation}
Thus, one may represent each end-state \(S_j\) by multiplying the associated conditional event probabilities along its branch of the tree.

In this case study, each \(P\bigl[X_i=1\bigr]\) is assigned a (truncated) log-normal parameterization, reflecting the fact that event probabilities can span several orders of magnitude. Let \(\mu_i,\sigma_i\) denote the log-space mean and standard deviation of event \(X_i\). Under truncation rules (e.g., restricting \(\mu_i\in[10^{-10},1]\) and \(\sigma_i\in[10^{-10},10^{4}]\)), the resulting probability stays in \((0,1)\) and avoids extreme instabilities. These are plotted in Figure \ref{fig:conditionals_initial} as normalized kernel density estimates\cite{terrell_variable_1992}.

\subsection{Loss Function Definition}
Given a set of target or observed end-state frequencies \(\{p_{j}^{\mathrm{obs}}\}_{j=1}^m\), the task is to infer \(\{\mu_i,\sigma_i\}_{i=1}^n\) so that the predicted frequencies 
\[
p_{j}^{\mathrm{pred}}\;\equiv\;P\bigl(S_j \mid \{\mu_i,\sigma_i\}\bigr)
\]
match \(p_{j}^{\mathrm{obs}}\) as closely as possible. Denoting \(\boldsymbol{\theta}=(\mu_1,\sigma_1,\dots,\mu_n,\sigma_n)\) for all events, the optimal parameters solve a constrained minimization:
\begin{equation}
\label{eq:ebr2_optimization}
\min_{\boldsymbol{\theta}\in\Omega}
\quad
\mathcal{L}\bigl(\boldsymbol{\theta};\,\{p_{j}^{\mathrm{obs}}\}\bigr)
\quad
\text{subject to truncation and system constraints,}
\end{equation}
where \(\Omega\) encodes bounds (e.g., \(\mu_i,\sigma_i \ge 10^{-10}\)), and \(\mathcal{L}\) is a loss function. Here, one defines \(\mathcal{L}\) via a Normalized Relative Logarithmic Error (NRLE), which balances discrepancies in both the predicted end-state frequencies and the tails of the distributions. A simplified version of NRLE is:
\begin{equation}
\label{eq:ebr2_nrle}
\text{NRLE} 
\;=\;
\frac{1}{m}
\sum_{j=1}^{m}
\;\frac{1}{2}
\Bigl(
  \text{MAE}\bigl(\log[p_{j}^{\mathrm{obs}}+\epsilon],\,\log[p_{j}^{\mathrm{pred}}+\epsilon]\bigr)
  \;+\;
  \text{MAE}\bigl(\sigma_{j}^{\mathrm{obs}},\,\sigma_{j}^{\mathrm{pred}}\bigr)
\Bigr),
\end{equation}
where \(\text{MAE}\) denotes mean absolute error, and \(\epsilon\) is a small positive constant to avoid \(\log(\,0\,)\). The terms \(\sigma_{j}^{\mathrm{obs}}\) and \(\sigma_{j}^{\mathrm{pred}}\) refer to log-space standard deviations for the respective distributions of (or mapped from) end-states or functional events. By design, this objective penalizes deviations of both central tendencies and spread. A gradient-based algorithm (e.g., Adam \citep{zhang_improved_2018}) then iteratively refines \(\{\mu_i,\sigma_i\}\), using automatic differentiation with respect to \(\mathcal{L}\).

\subsection{Results \& Discussion}
\label{sec:results_and_discussion_estimation}
The parameter estimation recovered target distributions and corresponding end-state frequencies with near-accurate fidelity, indicating that the method is capable of approximating underlying probabilities from limited inputs. The predicted end-state frequency estimates are plotted in Figure \ref{fig:end-states_estimated}.

Specifically, end-state frequencies estimated under the constrained optimization process diverged from reported references by small margins: on average, the mean values were recovered with an error of about \((1.08\pm 0.96)\%\), the 5th percentile with \((4.39\pm 7.09)\%\), and the 95th percentile with \((3.82\pm 5.91)\%\).  Such deviations suggest that the overall approach captures the central tendencies of event probabilities reasonably well, while still exhibiting moderate scatter in both lower and upper distribution tails.  Recurrence of larger discrepancies in selected events (e.g., certain fire detection or suppression paths) emphasizes the known difficulty of accurately modeling rare failure or success probabilities—particularly when the choice of distribution (e.g., log-normal) imposes strong structural assumptions on the shapes of these probability curves.

Despite these promising quantitative metrics, two issues warrant discussion. First, although end-state frequencies are reproduced within small mean errors, there is a real possibility of overfitting to the specified targets.  The optimization-driven procedure can finely tune parameters to minimize a chosen loss function; however, doing so may lead to calibrated event probabilities that reflect artifacts of the objective rather than a physically robust representation.  This risk is heightened when dealing with low-probability events (e.g., a rare liquid metal fire condition combined with other system failures)—situations that often exhibit limited empirical data.  

Second, the truncation and bounds on the log-normal parameterization, while necessary for numerical stability, can restrict the feasible solution space in unintended ways.  Large or extremely small event probabilities, particularly in tail regions, must fit within these truncated distributions.  If the true system behavior lies outside the assumed bounds, the resulting estimates may systematically under- or overestimate important tail events.  This possibility is underscored by the modest underestimation observed at the 95th percentile for certain functional events in the demonstration.
\begin{landscape}
\begin{figure}[ht!]
  \centering
\includegraphics[width=\textwidth]{parts/4_learning/1_param/figs/conditional_events_prior.png}
    \caption{Initial vs Target Functional Event Probability Distributions}
    \label{fig:conditionals_initial}
\end{figure}

\begin{figure}[hb!]
\centering
\includegraphics[width=\textwidth]{parts/4_learning/1_param/figs/end_states_prior.png}
    \caption{Initial vs Target End-State Frequency Distributions}
     \label{fig:end-states_initial}
\end{figure}
\end{landscape}

\clearpage
\begin{landscape}
\begin{table}[ht!]
\centering
\caption{Estimated vs Target Functional Event Probabilities Summarized}
\label{tab:conditional_estimated}
\scriptsize
\sisetup{table-format=1.2e-2}
\begin{tabular}{
    l
    S[table-format=1.2e-2]
    S[table-format=1.2e-2]
    S[table-format=1.2e-2]
    S[table-format=1.2e-2]
    S[table-format=1.2e-2]
    S[table-format=1.2e-2]
    S[table-format=1.2e-2]
    S[table-format=1.2e-2]
    S[table-format=1.2e-2]
    }
\toprule
\multirow{2}{*}{Event} & \multicolumn{3}{c}{$5^{th}$ Percentile} & \multicolumn{3}{c}{Mean} & \multicolumn{3}{c}{$95^{th}$ Percentile} \\
\cmidrule(lr){2-4} \cmidrule(lr){5-7} \cmidrule(lr){8-10}
& {Estimated} & {Target} & {Error\footnotemark} & {Estimated} & {Target} & {Error\footnotemark[\value{footnote}]} & {Estimated} & {Target} & {Error\footnotemark[\value{footnote}]} \\
\midrule
SDFR & 3.28e-03 & 3.30e-03 & -2.30e-05 & 4.24e-03 & 4.30e-03 & -5.97e-05 & 5.37e-03 & 5.47e-03 & -1.07e-04 \\
LMFD & 4.39e-08 & 4.29e-08 & 9.85e-10  & 1.00e-06 & 9.99e-07 & 7.13e-10  & 3.70e-06 & 3.70e-06 & -5.09e-09 \\
RFIR & 2.49e-06 & 2.39e-06 & 9.28e-08  & 4.13e-06 & 4.00e-06 & 1.29e-07  & 6.32e-06 & 6.15e-06 & 1.71e-07  \\
LLRF & 4.73e-03 & 4.72e-03 & 1.94e-05  & 9.93e-03 & 9.99e-03 & -6.36e-05 & 1.77e-02 & 1.79e-02 & -2.25e-04 \\
$\text{SSSD} \mid \overline{\text{LLRF}} $\footnotemark[2] & 8.72e-03 & 8.35e-03 & 3.68e-04  & 1.01e-02 & 1.00e-02 & 5.37e-05  & 1.15e-02 & 1.19e-02 & -3.31e-04 \\
$\text{SSSD} \mid \text{LLRF} $ & 4.93e-01 & 4.06e-01 & 8.74e-02  & 4.94e-01 & 5.00e-01 & -5.48e-03 & 4.96e-01 & 6.07e-01 & -1.11e-01 \\
$\text{SYSO} \mid \overline{\text{LLRF}} $ & 8.24e-05 & 8.33e-05 & -9.54e-07 & 1.36e-04 & 1.35e-04 & 8.31e-07  & 2.07e-04 & 2.03e-04 & 3.85e-06  \\
$\text{SYSO} \mid \text{LLRF} $ & 8.54e-02 & 8.60e-02 & -6.49e-04 & 9.89e-02 & 1.00e-01 & -1.11e-03 & 1.14e-01 & 1.15e-01 & -1.66e-03 \\
\bottomrule
\end{tabular}
\end{table}

\begin{figure}[hb!]
  \centering
\includegraphics[width=\textwidth]{parts/4_learning/1_param/figs/conditional_events_predicted.png}
    \caption{Estimated vs Target Functional Event Probability Distributions}
    \label{fig:conditional_estimated}
\end{figure}

\footnotetext[1]{[Estimated - Target], negative values represent underestimates.}
\footnotetext[2]{$ A \mid \overline{B} $: event A conditional on the non-occurrence of event B.}
\end{landscape}
\clearpage

\clearpage
\begin{landscape}

\begin{table}[ht!]
\centering
\caption{Estimated vs Target End-State Frequencies Summarized}
\label{tab:summary}
\scriptsize
\sisetup{table-format=1.2e-2}
\begin{tabular}{
    l
    S[table-format=1.2e-2]
    S[table-format=1.2e-2]
    S[table-format=1.2e-2]
    S[table-format=1.2e-2]
    S[table-format=1.2e-2]
    S[table-format=1.2e-2]
    S[table-format=1.2e-2]
    S[table-format=1.2e-2]
    S[table-format=1.2e-2]
    }
\toprule
\multirow{2}{*}{Event} & \multicolumn{3}{c}{$5^{th}$ Percentile} & \multicolumn{3}{c}{Mean} & \multicolumn{3}{c}{$95^{th}$ Percentile} \\
\cmidrule(lr){2-4} \cmidrule(lr){5-7} \cmidrule(lr){8-10}
& {Estimated} & {Target} & {Error\footnotemark[3]} & {Estimated} & {Target} & {Error\footnotemark[\value{footnote}]} & {Estimated} & {Target} & {Error\footnotemark[\value{footnote}]} \\
\midrule
SDFR-0\footnotemark[4] & 9.95e-01 & 9.95e-01 & 1.08e-04 & 9.96e-01 & 9.96e-01 & 5.98e-05 & 9.97e-01 & 9.97e-01 & 2.30e-05 \\
SDFR-1 & 3.21e-03 & 3.23e-03 & -2.26e-05 & 4.16e-03 & 4.22e-03 & -5.86e-05 & 5.27e-03 & 5.37e-03 & -1.06e-04 \\
SDFR-2 & 3.12e-05 & 3.07e-05 & 5.81e-07 & 4.22e-05 & 4.26e-05 & -3.56e-07 & 5.52e-05 & 5.69e-05 & -1.70e-06 \\
SDFR-3 & 3.15e-09 & 3.15e-09 & -7.36e-12 & 5.73e-09 & 5.74e-09 & -1.31e-11 & 9.33e-09 & 9.35e-09 & -2.39e-11 \\
SDFR-4 & 9.62e-06 & 9.22e-06 & 4.07e-07 & 2.13e-05 & 2.15e-05 & -1.90e-07 & 3.94e-05 & 4.07e-05 & -1.33e-06 \\
SDFR-5 & 8.47e-06 & 8.31e-06 & 1.66e-07 & 1.88e-05 & 1.94e-05 & -5.87e-07 & 3.47e-05 & 3.67e-05 & -2.04e-06 \\
SDFR-6 & 9.10e-07 & 9.07e-07 & 3.35e-09 & 2.06e-06 & 2.15e-06 & -9.10e-08 & 3.85e-06 & 4.13e-06 & -2.86e-07 \\
SDFR-7 & 9.84e-09 & 9.58e-09 & 2.64e-10 & 1.75e-08 & 1.72e-08 & 3.08e-10 & 2.82e-08 & 2.79e-08 & 3.17e-10 \\
SDFR-8 & 1.81e-10 & 1.80e-10 & 1.85e-12 & 4.18e-09 & 4.24e-09 & -5.50e-11 & 1.56e-08 & 1.58e-08 & -2.53e-10 \\
\bottomrule
\end{tabular}
\end{table}

\begin{figure}[ht!]
\centering
\includegraphics[width=\textwidth]{parts/4_learning/1_param/figs/end_states_predicted.png}
    \caption{Estimated vs Target End-State Frequency Distributions}
    \label{fig:end-states_estimated}
\end{figure}

\footnotetext[3]{[Estimated - Target], negative values represent underestimates.}
\footnotetext[4]{The likelihood of no SDFR. Computed by subtracting all end-state frequencies from the total probability.}
\end{landscape}
\clearpage

