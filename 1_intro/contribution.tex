\section{Key Contributions}
The principal technical contributions are:

\begin{enumerate}[label=\textbf{C\arabic*}.,leftmargin=2em]
  \item \textbf{Unified, data-parallel Monte-Carlo framework.}  We design mcSCRAM, a hardware-accelerated sampler that evaluates the \emph{entire} probabilistic directed acyclic graph (PDAG)—all coupled event trees and fault trees—in a single, layered pass without minimal-cut enumeration.

  \item \textbf{Hardware-native voting and cardinality gates.}  We introduce bit-parallel algorithms for $k$-of-$n$, at-most, exact, and cardinality predicates, prove estimator equivalence to classical AND/OR expansions, and show exponential savings in graph size and kernel-launch overhead.

  \item \textbf{Knowledge compilation re-imagined for Monte Carlo queries.}  We interrogate the design space of circuit transformations when the end-user query is high-throughput sampling rather than tractable logical inference.  Dispensing with traditional normal-form constraints (e.g.\ decomposability and smoothness), we develop a pipeline that prioritizes arithmetic intensity, graph depth, and kernel locality; an indexed linear map and iterative gate normalization drive up faster compilation and a median $1.3\times$ structural compression, illustrating the different optimality criteria that emerge once exact inference is no longer the goal.

  \item \textbf{Composite convergence diagnostics.}  We formulate a stopping rule that fuses Wald, Bayesian, and information-theoretic criteria, providing rigorous error guarantees even in the rare-event regime and under external wall-time or iteration budgets.

  \item \textbf{Domain-specific extensions.}  The framework incorporates: (i) common-cause failure groups through auxiliary/shadow nodes, (ii) first-order importance and sensitivity measures via in-tally covariance accumulation, and (iii) an importance-sampling module for ultra-rare events—all without modifying the core kernels.

  \item \textbf{Open-source implementation and benchmarks.}  A SYCL-based reference implementation runs on GPUs, CPUs, and FPGAs; public benchmarks on the 43-model Aralia dataset demonstrate sub-percent relative error on graphs with a few thousand events in <5 s on entry-level GPUs, surpassing established PRA tools in throughput while matching their accuracy.
\end{enumerate}