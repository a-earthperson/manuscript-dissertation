\subsection{Results \& Discussion}
\label{sec:results_and_discussion_estimation}
The parameter estimation recovered target distributions and corresponding end-state frequencies with near-accurate fidelity, indicating that the method is capable of approximating underlying probabilities from limited inputs. The predicted end-state frequency estimates are plotted in Figure \ref{fig:end-states_estimated}.

Specifically, end-state frequencies estimated under the constrained optimization process diverged from reported references by small margins: on average, the mean values were recovered with an error of about \((1.08\pm 0.96)\%\), the 5th percentile with \((4.39\pm 7.09)\%\), and the 95th percentile with \((3.82\pm 5.91)\%\).  Such deviations suggest that the overall approach captures the central tendencies of event probabilities reasonably well, while still exhibiting moderate scatter in both lower and upper distribution tails.  Recurrence of larger discrepancies in selected events (e.g., certain fire detection or suppression paths) emphasizes the known difficulty of accurately modeling rare failure or success probabilities—particularly when the choice of distribution (e.g., log-normal) imposes strong structural assumptions on the shapes of these probability curves.

Despite these promising quantitative metrics, two issues warrant discussion. First, although end-state frequencies are reproduced within small mean errors, there is a real possibility of overfitting to the specified targets.  The optimization-driven procedure can finely tune parameters to minimize a chosen loss function; however, doing so may lead to calibrated event probabilities that reflect artifacts of the objective rather than a physically robust representation.  This risk is heightened when dealing with low-probability events (e.g., a rare liquid metal fire condition combined with other system failures)—situations that often exhibit limited empirical data.  

Second, the truncation and bounds on the log-normal parameterization, while necessary for numerical stability, can restrict the feasible solution space in unintended ways.  Large or extremely small event probabilities, particularly in tail regions, must fit within these truncated distributions.  If the true system behavior lies outside the assumed bounds, the resulting estimates may systematically under- or overestimate important tail events.  This possibility is underscored by the modest underestimation observed at the 95th percentile for certain functional events in the demonstration.