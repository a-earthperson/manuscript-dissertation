\section{Bitpacked Random Number Generator}

Monte Carlo simulations, probability evaluations, and other sampling-based procedures benefit greatly from efficient, high-quality random number generators (RNGs). A large class of modern RNGs are known as \textit{counter-based PRNGs}, because they use integer counters (e.g., 32-bit or 64-bit) along with a stateless transformation to produce random outputs. The \emph{Philox} family of counter-based PRNGs is a well-known example, featuring fast generation, high period, and good statistical properties. In this section, we discuss the general principles of counter-based PRNGs, explain how Philox fits into this paradigm, analyze its complexity, and present a concise pseudocode version of the \(\text{Philox }4\times32\text{-10}\) variant. Subsequently, we detail the bitpacking scheme used to reduce memory consumption when storing large numbers of Bernoulli samples.

A counter-based PRNG maps a user-supplied \emph{counter} (plus, optionally, a \emph{key}) to a fixed-size block of random bits via a deterministic function. Formally, if 
\[
  \mathbf{x} \;=\; (x_1, x_2, \ldots, x_k)
\]
is a vector of one or more 32-bit or 64-bit counters, and 
\[
  \mathbf{k} \;=\; (k_1, k_2, \ldots, k_m)
\]
is a key vector, then a counter-based PRNG defines a transformation 
\[
   \mathcal{F}(\mathbf{x}, \mathbf{k})
   \;=\;
   (\rho_1, \rho_2, \ldots, \rho_r),
\]
where each \(\rho_j\) is typically a 32-bit or 64-bit output. Different increments of the counter \(\mathbf{x}\) produce different pseudo-random outputs \(\rho_j\). The process is stateless in the sense that advancing the RNG amounts to incrementing the counter (e.g., \(\mathbf{x}\mapsto \mathbf{x} + 1\)).

Compared to older recurrence-based RNGs such as linear congruential generators or the Mersenne Twister, counter-based methods offer more straightforward parallelization, reproducibility across multiple streams, and strong structural simplicity: no internal state must be updated or maintained. This is particularly valuable in distributed Monte Carlo simulations or GPU-based sampling, where each thread or work-item can be assigned a different counter. Philox constructs its pseudo-random outputs by applying a small set of mixed arithmetic (multiplication/bitwise) rounds to an input \emph{counter} plus \emph{key}. In particular, \(\mathrm{Philox}\,4\times32\text{-10}\) (often shortened to “Philox-4x32-10”) works on four 32-bit integers at a time:
\[
  \mathbf{S} = (S_0, S_1, S_2, S_3),
  \qquad
  \mathbf{K} = (K_0, K_1).
\]
The four elements \(\{S_0, S_1, S_2, S_3\}\) collectively represent the counter, e.g., \((x_0, x_1, x_2, x_3)\). The two key elements \((K_0, K_1)\) are used to tweak the generator’s sequence. A single invocation of Philox-4x32-10 transforms \(\mathbf{S}\) into four new 32-bit outputs after ten rounds of mixing. At each round, the algorithm:
\begin{enumerate}
    \item Multiplies two of the state words by fixed “magic constants” to create partial products.
    \item Takes the high and low 32-bit portions of those 64-bit products.
    \item Incorporates the round key to shuffle the words.
    \item Bumps the key by adding constant increments \((\mathrm{W32A} = 0x9E3779B9 \text{ and } \mathrm{W32B} = 0xBB67AE85)\).
\end{enumerate}
After ten rounds, the final \((S_0, S_1, S_2, S_3)\) is returned as the pseudo-random block. A new call to Philox increases the counter \(\mathbf{S}\) by one (e.g., \(S_3 \mapsto S_3 + 1\)) and re-enters the same function. The Philox-4x32-10 algorithm is designed so that each blocking call requires a \emph{constant number} of operations, independent of the size of any prior “state.” Specifically, each round involves:
\[
  \mathcal{O}(1)\;\text{ arithmetic operations},
\]
and there are \(\mathrm{R} = 10\) rounds. Thus, each Philox invocation is asymptotically constant time \(\mathcal{O}(\mathrm{R}) = \mathcal{O}(1)\). The total cost to generate 128 bits (4 words \(\times\) 32 bits) is therefore constant time per call.

\subsection{Philox-4x32-10 Pseudocode}
Our implementation follows the standard 10-round approach for generating one block of four 32-bit random words, also called Philox-4x32-10. Let \(M_{\mathrm{A}}=0xD2511F53\), \(M_{\mathrm{B}}=0xCD9E8D57\) be the multipliers, and let \((K_0, K_1)\) be the key which is updated each round by \(\mathrm{W32A}=0x9E3779B9\) and \(\mathrm{W32B}=0xBB67AE85\). The function \(\text{Hi}(\cdot)\) returns the high 32 bits of a 64-bit product, and \(\text{Lo}(\cdot)\) returns the low 32 bits. Because each call produces four 32-bit pseudo-random words, Philox-4x32-10 is particularly convenient for batched sampling. If only a single 32-bit word is needed, one can still call the function and discard the excess words; however, many applications consume all four outputs (e.g., to produce four floating-point variates).

% \begin{algorithm}[ht]
% \caption{Philox-4x32-10}
% \SetKwInOut{Input}{Input}
% \SetKwInOut{Output}{Output}

% \Input{Four 32-bit counters \((S_0, S_1, S_2, S_3)\), Key \((K_0, K_1)\)}
% \Output{Transformed counters \((S_0, S_1, S_2, S_3)\)}

% \SetKwFunction{FPhiloxRound}{Philox\_Round}
% \SetKwFunction{FPhilox}{Philox4x32\_10}

% \SetKwProg{Fn}{Function}{:}{}
% \Fn{\FPhiloxRound{$(S_0, S_1, S_2, S_3), (K_0, K_1)$}}{
%   $P_0 \gets M_{\mathrm{A}} \times S_0$\;  \tcp{64-bit product}
%   $P_1 \gets M_{\mathrm{B}} \times S_2$\;  \tcp{64-bit product}
%   $T_0 \gets \text{Hi}(P_1)\;\oplus\;S_1\;\oplus\;K_0$\;
%   $T_1 \gets \text{Lo}(P_1)$\;
%   $T_2 \gets \text{Hi}(P_0)\;\oplus\;S_3\;\oplus\;K_1$\;
%   $T_3 \gets \text{Lo}(P_0)$\;
%   $K_0 \gets K_0 + \mathrm{W32A}$\;
%   $K_1 \gets K_1 + \mathrm{W32B}$\;
%   \KwRet{$(T_0, T_1, T_2, T_3), (K_0, K_1)$}\;
% }
% \Fn{\FPhilox{$(S_0, S_1, S_2, S_3), (K_0, K_1)$}}{
%   \For{$i \gets 1$ \KwTo $10$}{
%     $(S_0, S_1, S_2, S_3), (K_0, K_1) \gets$ \FPhiloxRound{$S_0, S_1, S_2, S_3, K_0, K_1$}\;
%   }
%   \KwRet{$(S_0, S_1, S_2, S_3)$}\;
% }
% \end{algorithm}

\subsection{Bitpacking for Probability Sampling}
It takes exactly one bit to represent the outcome of a trial. If these If outcomes are stored naively, each one occupies a full 8-bit byte. Hence, only \( \tfrac{1}{8} \) of the allocated space is used for actual data. By instead packing up to \(w\) indicators into a \(w\)-bit machine word, the memory usage can be reduced by a factor of up to \(8\) (in the simplest scenario of 8-bit groupings). In more general terms:
\[
  \text{Memory usage }M_{\text{naive}}
  \;=\;
  N \times 8\;\text{bits},
  \qquad
  \text{Memory usage }M_{\text{pack}}
  \;=\;
  \left\lceil\frac{N}{w}\right\rceil \,\times\,w\;\text{bits}.
\]
In our implementation, each call to Philox-4x32-10 yields 128 bits of randomness. We use those bits to draw exactly 128 Bernoulli outcomes at once, then combine them into a \(\mathrm{bitpack}\) of two 64-bit integers. For instance, if we choose a batch size of \(4\)-bits to represent four Bernoulli samples in a single chunk, we can:

\begin{enumerate}
    \item Generate a block \(\{r_0, r_1, r_2, r_3\}\) of four 32-bit random integers from Philox.
    \item Convert each \(r_i\) into a uniform \([0,1)\) floating-point value by dividing by \(2^{32}\).
    \item Compare each to the target probability \(p\).
    \item Form a 4-bit integer, each bit set to \(1\) if the corresponding comparison succeeded, or \(0\) otherwise.
\end{enumerate}

\noindent Repeating these steps for multiple rounds of 4 bits each can fill a 16-bit or 32-bit \(\mathrm{bitpack}\) variable with many Bernoulli indicators. Then it can be stored into an array at a single index, reducing memory overhead by constant factor of $N$. 


% \begin{algorithm}[H]
% \caption{Bitpacking of Four Bernoulli Samples Into a 4-Bit Block}
% \SetKwInOut{Input}{Input}
% \SetKwInOut{Output}{Output}

% \Input{Probability \(p \in [0,1]\), 4 random 32-bit words \((r_0, r_1, r_2, r_3)\)}
% \Output{4-bit integer \(\mathrm{bits}\), storing 4 Bernoulli draws}

% \SetKwFunction{FMain}{FourBitPack}
% \SetKwProg{Fn}{Function}{:}{}
% \Fn{\FMain{\(p, (r_0, r_1, r_2, r_3)\)}}{
%     $\mathrm{bits} \gets 0$\;
%     \For{$i \gets 0$ \KwTo $3$}{
%         $u_i \gets \frac{r_i}{2^{32}} \;\; \in [0,1)$\;
%         $b_i \gets \begin{cases}
%             1, & \text{if }u_i < p\\
%             0, & \text{else}
%         \end{cases}$\;
%         $\mathrm{bits} \gets \mathrm{bits} \;\;\Vert\;\; (b_i \ll i)$ \tcp*[r]{Set bit \(i\) to \(b_i\)}
%     }
%     \KwRet $\mathrm{bits}$\;
% }
% \end{algorithm}

\noindent In this procedure, \(\Vert\) denotes a bitwise OR, and \(\ll\) denotes a left shift. One then repeats the above call to accumulate multiple 4-bit blocks (e.g., for a total of 16 bits, one calls FourBitPack four times and merges the results with the appropriate shifts).