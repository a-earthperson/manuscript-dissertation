\section{PRA Models as Directed Acyclic Graphs}
\label{sec:unified_pra_dag}

Up to this point, we have introduced ETs to capture the forward evolution of scenarios and FTs to capture the top-down decomposition of system failures. In a full-scale PRA, many ETs and FTs are linked to form a single overarching model. The goal is to represent:
\begin{enumerate}
\item The branching structure of multiple event trees, which may feed into one another (ET\(\to\)ET),  
\item Multiple fault trees that themselves can reference or be referenced by other FTs (FT\(\to\)FT),  
\item Event trees that invoke fault trees to quantify key failure probabilities (ET\(\to\)FT), and similarly fault trees whose outcome may direct the next branch or state in an event tree.  
\end{enumerate}
All of these interconnections can be consolidated into a single DAG.  In broad terms, its nodes stand for either (i)~ET nodes (initiating or functional events), (ii)~FT gates or intermediate events, or (iii)~basic events.

\subsection{Basic Structure and Notation}
Let us denote:
\begin{itemize}
\item \(\{\Gamma_1, \Gamma_2, \dots, \Gamma_{M}\}\) as the \emph{collection of event trees}, where each \(\Gamma_{i}\) may represent a different initiating event or system phase.  Every \(\Gamma_i\) is itself structured as in Section~\ref{sec:event_tree_definition}, with a set of node events and directed edges (success/failure branches).  
\item \(\{\Phi_{1}, \Phi_{2}, \dots, \Phi_{N}\}\) as the \emph{collection of fault trees}, each built according to Section~\ref{sec:fault_tree_definition}.  Every \(\Phi_j\) has a unique top event, an acyclic arrangement of gates (AND, OR, voting, etc.), and a set of basic events.  
\end{itemize}

In a large PRA, any ET \(\Gamma_i\) may:
\begin{itemize}
\item Lead to another ET \(\Gamma_{k}\) under certain branch outcomes (ET\(\to\)ET).  
\item Include a branch that requires computing ``System \(X\) fails'' via a fault tree \(\Phi_j\) (ET\(\to\)FT).  
\end{itemize}
Similarly, a fault tree \(\Phi_{j}\) may:
\begin{itemize}
\item Contain the top event of another fault tree \(\Phi_{k}\) as one of its inputs (FT\(\to\)FT).  
\item Generate an outcome (e.g.\ subsystem fails) that triggers a branch in some event tree \(\Gamma_i\).  
\end{itemize}
These inter-dependencies can be organized into a single DAG, denoted
\[
\mathcal{M} \;=\; \bigl(\mathcal{V},\,\mathcal{A}\bigr),
\]
where:
\begin{itemize}
\item \(\mathcal{V}\) is the set of \emph{all} nodes in the unified model.  Each node \(v\in \mathcal{V}\) has a type indicating whether it belongs to an event tree (ET-node), a fault tree (FT-gate), or is a basic event (BE).  
\item \(\mathcal{A}\subseteq \mathcal{V}\times \mathcal{V}\) is the set of directed edges.  Each \((u,v)\in \mathcal{A}\) signifies a logical or probabilistic dependency from node \(u\) to node \(v\).  
\end{itemize}
By design, \(\mathcal{M}\) is acyclic: no path loops back through the same node.  This condition prevents paradoxical definitions of probabilities or statements (e.g., an event that depends on itself).

\subsection{Nodes and Their Inputs}

We partition \(\mathcal{V}\) into three principal categories:
\begin{enumerate}
\item \textbf{Basic Events (BEs).}  Let \(\mathcal{B}\subset \mathcal{V}\) be the set of all basic events across all FTs.  Each \(b\in\mathcal{B}\) is associated with a failure probability \(p(b)\in [0,1]\).  A node \(b\) in the DAG has no incoming edges (i.e., it is a leaf source for the rest of the logic).  

\item \textbf{Fault Tree (FT) Nodes.}  Let \(\mathcal{G}\subset \mathcal{V}\) be the set of \emph{internal FT-gates or intermediate FT-events}\footnote{In some treatments, all FT internal nodes are called gates, even if no actual gate symbol is used.  We adopt the broader term \emph{FT~nodes} for uniformity.}, unified across \(\{\Phi_1,\dots,\Phi_N\}\).  
    \begin{itemize}
    \item Each FT node \(g\in \mathcal{G}\) has one output event \(g\) (the node itself in the DAG).  
    \item The set of inputs to \(g\) may include basic events \(\mathcal{B}\) (e.g., component failures) and/or other FT nodes \(\mathcal{G}\) (subsystem-level events). If \(g\) is the top event of \(\Phi_j\), then it may appear as an input into a \emph{different} FT’s gate or an ET node.  
    \item As with standard FT logic, each gate has a type in \(\{\mathrm{AND},\,\mathrm{OR},\,\mathrm{VOT}(k/n),\,\ldots\}\).  Its output (failure state) is a Boolean function of its inputs’ failure states.
    \end{itemize}

\item \textbf{Event Tree (ET) Nodes.}  Let \(\mathcal{E}\subset \mathcal{V}\) be the set of \emph{ET-nodes}, each referring either to an initiating event (IE) or to a functional event within some event tree \(\Gamma_i\).  
    \begin{itemize}
    \item An ET node \(e\) in \(\Gamma_i\) can have multiple outgoing edges, each labeled by a particular outcome (e.g., success/failure).  These edges may lead to another ET node (continuing the same event tree), to the root/top event of a fault tree (to evaluate subsystem reliability), or even to the root of a \emph{different} event tree (ET\(\to\)ET).
    \item As in Section~\ref{sec:event_tree_definition}, each outgoing branch from \(e\) has an associated conditional probability, conditioned on the event \(e\) itself having occurred.  
    \end{itemize}
\end{enumerate}
Hence, 
\[
\mathcal{V} \;=\;
\underbrace{\mathcal{B}}_{\text{basic events}}
\;\;\cup\;\;
\underbrace{\mathcal{G}}_{\text{FT nodes}}
\;\;\cup\;\;
\underbrace{\mathcal{E}}_{\text{ET nodes}},
\]
and every node \(v\) in \(\mathcal{V}\) has an \emph{input set}
\[
I(v)
\;\;\subseteq\;\;
\mathcal{B}\,\cup\,\mathcal{G}\,\cup\,\mathcal{E}
\;\;=\;\;\mathcal{V}\setminus\{v\},
\]
indicating which nodes feed into \(v\).  By the DAG property, \(v\) cannot be an ancestor of itself.

\subsection{Edge Types and Probability Assignments}

Each directed edge \((u,v)\in \mathcal{A}\) belongs to one of several categories:

\begin{itemize}
\item \textbf{ET \(\to\) ET edges. [Transfers]}  These edges connect an ET node \(u\in\mathcal{E}\) to another ET node \(v\in\mathcal{E}\) within the same tree \(\Gamma_i\) or leading to a subsequent tree \(\Gamma_{k}\).  In typical diagrams, these edges denote if \(u\) occurs, then with probability \(\theta_{u\to v}\) we transition to \(v\).  Probabilities on all child edges of \(u\) sum to 1, reflecting the partition of possible outcomes.

\item \textbf{ET \(\to\) FT edges. [Functional Events]}  These edges represent the case where an ET node \(u\in\mathcal{E}\) designates Check if subsystem \(\Phi_j\) has failed.  Formally, the next node \(v\in\mathcal{G}\) is the top event (or relevant subsystem event) in fault tree \(\Phi_j\).  The probability of \(v\) failing is not assigned directly on the edge but is computed via the logical structure of \(\Phi_j\).  

\item \textbf{FT \(\to\) FT edges. [Transfer Gates]}  Such edges arise when the top event (or an intermediate gate) \(u\in\mathcal{G}\) of one fault tree is input to a gate \(v\in\mathcal{G}\) in another fault tree.  For instance, if \(\Phi_{1}\) captures the failure mode of a pump and \(\Phi_{2}\) captures the failure of a coolant subsystem that includes that same pump’s top event.  

\item \textbf{FT \(\to\) ET edges. [Initiating Events]}  Less common but still possible are edges that carry an outcome of a fault tree node \(u\in\mathcal{G}\) to an ET node \(v\in\mathcal{E}\).  For instance, an initiating event might depend on whether a certain subsystem fails, as computed by a separate FT \(\Phi_j\). Support system FTs are one such example.

\item \textbf{BEs as sources (no incoming edges).}  Each basic event \(b\in\mathcal{B}\) has probability \(p(b)\) of failing, so \(\Pr[X_b=1]\!=\!p(b)\).  These do not have incoming edges because they represent fundamental failure modes, not dependent on other events within the model.  
\end{itemize}
Denote by \(\theta_{u\to v}\) the \emph{conditional probability} weighting an ET-type edge \((u\!\to\!v)\).  If node \(u\) splits into children \(v_1,\dots,v_k\), then
\[
\sum_{i=1}^k \theta_{u \to v_i} \;=\; 1,
\quad
\theta_{u \to v_i}\;\; \ge\;0.
\]
By contrast, FT-type edges do not carry numerical probabilities directly.  Instead, a gate node \(v\in \mathcal{G}\) aggregates its inputs’ \emph{fail/success states} via Boolean logic (AND, OR, \(\mathrm{VOT}(k/n)\), etc.) to yield \(\pi_{\mathcal{M}}(S,v)\in\{0,1\}\), the node \(v\)’s failure state under a set \(S\) of basic-event failures.

\subsection{Semantics of the Unified Model}

A full \emph{scenario} in \(\mathcal{M}\) extends from a designated \textit{initial node} (often an initiating event \(I\in \mathcal{E}\)) forward through whichever ET or FT edges are triggered.  Because no cycles exist, every path eventually terminates in either (a)~an ET leaf (end-state), (b)~a top event that is not expanded further, or (c)~a final subsystem outcome deemed not to propagate further risk.

\subsubsection{Failure States in the Fault Trees.}
For any subset \(S\subseteq \mathcal{B}\) of basic events that fail:
\begin{enumerate}
\item Each \(b\in \mathcal{B}\) fails iff \(b\in S\).  
\item Each FT node \(g\in \mathcal{G}\) has failure state \(\pi_{F}(S,g)\) determined by the usual fault tree semantics (Section~\ref{sec:fault_tree_definition}).  
\end{enumerate}
That is, a node \(g\) in a fault tree \(\Phi_j\) is in failure mode when its logical gate type indicates failure is activated by the failures of its inputs (which might be other gates or basic events).

\subsubsection{Branching in the Event Trees.}
Whenever an ET node \(e\in \mathcal{E}\) is reached, outgoing edges \(\{(e \to e_1), (e\to e_2),\dots\}\) partially partition the scenario space.  The choice of which child \(e_i\) is realized is probabilistic, with probabilities \(\theta_{e\to e_i}.\)

\subsubsection{Event-Tree to Fault-Tree Links.}
If an edge \((e\to g)\) connects an ET node \(e\) to a \emph{fault tree top event} \(g\in \mathcal{G}\), the scenario path triggers the question Does \(g\) fail?  The probability that \(g\) is in failure, conditional on having arrived at node \(e\), is determined by the set \(S\subseteq \mathcal{B}\) of basic events that happen to fail in that scenario plus the gate logic of \(\Phi_j\).

Altogether, scenario outcomes in \(\mathcal{M}\) thus combine:
\begin{itemize}
\item \(\mathbf{X}_{\mathcal{B}} = \{\,X_b : b\in\mathcal{B}\}\), where \(X_b\in\{0,1\}\) indicates whether basic event \(b\) fails or not, and  
\item A chain of ET decisions or fault-tree outcomes, traveling through the DAG until reaching a terminal node.  
\end{itemize}
If all basic events are assumed independent with probabilities \(\{p(b)\}\), then the \emph{global} likelihood of a specific path \(\omega\) from an initiating event \(I\) to a final outcome (and with a particular pattern of success/failure across \(\mathcal{B}\)) factors into products of:
\begin{enumerate}
\item The product of \(\prod_{b\in S}p(b)\,\times\,\prod_{b\notin S}[\,1-p(b)\,]\) for the relevant \(S\subseteq\mathcal{B}\).  
\item The product of all ET-edge probabilities \(\theta_{u\to v}\) encountered along the path (\(u\in\mathcal{E}\)).  
\item The logical constraints from each visited fault tree node \(g\in \mathcal{G}\), which impose \(\pi_{F}(S,g)\in\{0,1\}\) in or out of failure.  
\end{enumerate}
Summing over all valid paths (or equivalently over all subsets \(S\subseteq \mathcal{B}\) and the result of each ET/FT branching) yields the total system risk measure, such as the probability of a severe radiological release, or the probability that a certain undesired top event emerges.

\subsection{Formal Definition of the Unified Model}

Bringing these elements together, we propose the following definition:

\begin{definition}[Unified PRA Model]
\label{def:unified_pra_dag}
A \emph{unified PRA model} is a directed acyclic graph
\[
\mathcal{M}
\;=\;
\Bigl\langle\,
  \mathcal{V}\;=\;\mathcal{B}\cup\mathcal{G}\cup\mathcal{E},
  \;\;\mathcal{A},
  \;\;p(\cdot),
  \;\;\pi_{F}
\Bigr\rangle
\]
with the following properties:

\begin{enumerate}
\item \(\mathcal{B}\) is the set of \textbf{basic events}, each \(b\in \mathcal{B}\) failing with probability \(p(b)\).  These nodes have no incoming edges in \(\mathcal{M}\).  
\item \(\mathcal{G}\) is the set of \textbf{FT nodes}, each representing a gate or intermediate event in a fault tree.  For \(g\in \mathcal{G}\), the function
\[
\pi_{F}(S,g)\;=\;\begin{cases}
1, & \text{if fault-tree logic declares \(g\) fails under \(S\subseteq\mathcal{B}\)},\\
0, & \text{otherwise}.
\end{cases}
\]
\item \(\mathcal{E}\) is the set of \textbf{ET nodes}, each having zero or more outgoing edges.  If node \(e\) has children \(\{v_1,\dots,v_k\}\subset \mathcal{V}\), then the edges \(\{(e\to v_i)\}_{i=1}^k\) carry probabilities \(\theta_{e\to v_i}\ge 0\) summing to 1.  
\item \(\mathcal{A}\subseteq (\mathcal{V}\times \mathcal{V})\) is the set of directed edges.  An edge \((u\to v)\) can be:
  \begin{itemize}
  \item \(\mathrm{ET\to ET}\):  event-tree branching,  
  \item \(\mathrm{ET\to FT}\):  an event tree referencing a fault tree’s top event,  
  \item \(\mathrm{FT\to FT}\):  linking one fault tree node to another’s input,  
  \item \(\mathrm{FT\to ET}\):  an FT outcome passed back to an event tree node,  
  \item \(\mathrm{BE}\to\varnothing\):  no edges emanate from a basic event node.  
  \end{itemize}
\item The graph \(\mathcal{M}\) is acyclic: no path in \(\mathcal{A}\) returns to a previously visited node.  
\end{enumerate}
\end{definition}

In practice, we anticipate large nuclear PRAs to contain hundreds of event trees and thousands of fault trees, sharing many basic events or subsystem-level fault trees.  By embedding them in \(\mathcal{M}\), one can systematically compute probabilities for any high-level risk measure (e.g., core damage, large release) by enumerating scenario paths or using specialized algorithms (e.g.\ binary decision diagrams, minimal cut set expansions, or simulation-based techniques).  The unified DAG structure codifies both the forward scenario expansions (ET logic) and the top-down sub-component dependencies (FT logic) without creating contradictions or cycles. Definition~\ref{def:unified_pra_dag} makes manifest \emph{which} event tree references \emph{which} fault tree, how fault trees are chained together, and how basic events ultimately feed every higher-level node.  Once built, one may:
\begin{itemize}
\item \textbf{Traverse} each path from an initiating event to a final end-state, propelling forward along the ET edges and evaluating any FT nodes via \(\pi_{F}\).
\item \textbf{Find minimal cut sets and path sets} by traversing the DAG and making cuts along the way.
\item \textbf{Sum} (or bound, or approximate) scenario probabilities to quantify overall risk.  
\item \textbf{Perform sensitivity analyses} by biasing subsets of basic-event probabilities \(p(b)\) or gate dependencies.
\end{itemize}
All standard PRA methods (minimal cut sets, Monte Carlo simulation, bounding formulas, etc.) remain applicable, but now from within a single unified DAG representation.