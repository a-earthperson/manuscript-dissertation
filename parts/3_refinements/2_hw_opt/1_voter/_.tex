\section{Efficient Voting Logic Evaluation using Bit Fields}


\subsection{Definition of k-of-n Voting Logic}
\label{sec:kn_voting_logic_definition}

A \emph{k-of-n} gate, denoted \(\mathrm{VOT}(k/n)\), outputs 1 if and only if \emph{at least} \(k\) of its \(n\) inputs equal 1. Such a gate, often referred to as a \emph{threshold} or \emph{majority voting} gate, conveniently models partial redundancy and majority-vote mechanisms. Concretely, let each of the \(n\) input events be represented by a binary variable:
\[
X_1, \, X_2, \, \dots, \, X_n \;\in\;\{0,1\}.
\]
The gate’s output \(Y\) is then defined by
\begin{equation}
\label{eq:kn_gate_boolean}
Y 
\;=\;
\begin{cases}
1, & \text{if }\, \displaystyle\sum_{i=1}^{n} X_i \;\ge\; k,\\[4pt]
0, & \text{otherwise}.
\end{cases}
\end{equation}
Equivalently, \(Y=1\) can be expressed as a disjunction of conjunctions:
\begin{equation}
\label{eq:k_of_n_or_of_ands}
Y 
\;=\;
\Bigl[\sum_{i=1}^{n} X_i \,\ge\, k\Bigr]
\;=\;
\bigvee_{\substack{S\,\subseteq\,\{1,\dots,n\}\\|S|=k}}
\;\;\biggl(\,\bigwedge_{i\in S} X_i\biggr),
\end{equation}
meaning that at least one subset \(S\subseteq \{1,\dots,n\}\) of size \(k\) has all its corresponding \(X_i\) set to 1. Any larger subset \(\lvert S\rvert > k\) naturally satisfies the same condition.

In many PRA tools, the \(\mathrm{VOT}(k/n)\) gate is provided as a built-in element. Internally, these tools may expand Eq.~\eqref{eq:k_of_n_or_of_ands} into an OR-of-ANDs or maintain a more compact representation (e.g., via binary decision diagrams). Under independence assumptions, standard failure-probability calculations proceed by summing over combinations of basic-event failures. However, an explicit expansion incurs the combinatorial factor
\[
\binom{n}{k} \;+\; \binom{n}{k+1} \;+\; \dots \;+\; \binom{n}{n},
\]
which can grow quickly as \(n\) increases.

\subsection{Probability Computation under Independence}
When each \(X_i\) is modeled as a Bernoulli random variable taking the value \(1\) with probability
\[
p_i \;=\; P(X_i = 1),
\]
assumed independent of the other inputs, one can write for each \(k\)-element subset \(S \subseteq \{1,\ldots,n\}\):
\[
A_S \;=\;\bigwedge_{i \in S} \{X_i = 1\},
\quad
P\bigl(A_S\bigr) 
\;=\;
\prod_{i \in S} p_i.
\]
The event \(Y=1\) in Eq.~\eqref{eq:k_of_n_or_of_ands} is the union of all such events \(A_S\) for \(\lvert S\rvert = k\). Hence, in principle, to compute
\[
P\Bigl(Y = 1\Bigr)
\;=\;
P\Bigl(\bigvee_{\lvert S\rvert = k} A_S\Bigr),
\]
one may apply the inclusion-exclusion principle. Specifically:
\begin{equation}
\label{eq:union_inclusion_exclusion}
P\Bigl(\bigvee_{\lvert S\rvert = k} A_S\Bigr)
\;=\;
\sum_{\lvert S\rvert=k}P(A_S)
\;-\;\sum_{1\le \alpha_1<\alpha_2\le M}P\Bigl(A_{S_{\alpha_1}}\cap A_{S_{\alpha_2}}\Bigr)
\;+\;\dots
\end{equation}
where \(M=\sum_{j=k}^{n} \binom{n}{j}\) is the total number of events \(A_S\) of interest. Unfortunately, direct enumeration of these intersections can become infeasible for large \(n\).

\subsubsection{Example of 3-of-5 Voting Logic}
\label{sec:3_of_5_voting_logic_example}

For example, consider the case where $k=3, n=5$. Using the notation from Section~\ref{sec:kn_voting_logic_definition}, the compressed form of the gate output is
\[
Y \;=\; \mathrm{VOT}(3/5)\bigl(X_1, X_2, X_3, X_4, X_5\bigr)
\;=\;
\bigl[\;X_1 + X_2 + X_3 + X_4 + X_5 \;\ge\; 3\bigr].
\]

\begin{figure}[h!]
    \centering
    \resizebox{\textwidth}{!}{%
    \usetikzlibrary {circuits.logic.US}
% \begin{tikzpicture}[circuit logic US,
%                     tiny circuit symbols,
%                     ]
%   \node[or]{\rotatebox{-90}{Y}}
%   child {
%     node[and]{}
%     child {
%       node[var] {\rotatebox{-90}{$X_3$}}
%     }
%     child {
%       node[var] {\rotatebox{-90}{$X_2$}}
%     }
%     child {
%       node[var] {\rotatebox{-90}{$X_1$}}
%     }
%   }
%   child {
%     node[and]{}
%     child {
%       node[var] {\rotatebox{-90}{$X_4$}}
%     }
%     child {
%       node[var] {\rotatebox{-90}{$X_2$}}
%     }
%     child {
%       node[var] {\rotatebox{-90}{$X_1$}}
%     }
%   }
%   child {
%     node[and]{}
%     child {
%       node[var] {\rotatebox{-90}{$X_3$}}
%     }
%     child {
%       node[var] {\rotatebox{-90}{$X_2$}}
%     }
%     child {
%       node[var] {\rotatebox{-90}{$X_1$}}
%     }
%   }
%   child {
%     node[and]{}
%     child {
%       node[var] {\rotatebox{-90}{$X_3$}}
%     }
%     child {
%       node[var] {\rotatebox{-90}{$X_2$}}
%     }
%     child {
%       node[var] {\rotatebox{-90}{$X_1$}}
%     }
%   }
%   child {
%     node[and]{}
%     child {
%       node[var] {\rotatebox{-90}{$X_3$}}
%     }
%     child {
%       node[var] {\rotatebox{-90}{$X_2$}}
%     }
%     child {
%       node[var] {\rotatebox{-90}{$X_1$}}
%     }
%   }
%   child {
%     node[and]{}
%     child {
%       node[var] {\rotatebox{-90}{$X_3$}}
%     }
%     child {
%       node[var] {\rotatebox{-90}{$X_2$}}
%     }
%     child {
%       node[var] {\rotatebox{-90}{$X_1$}}
%     }
%   }
%   child {
%     node[and]{}
%     child {
%       node[var] {\rotatebox{-90}{$X_3$}}
%     }
%     child {
%       node[var] {\rotatebox{-90}{$X_2$}}
%     }
%     child {
%       node[var] {\rotatebox{-90}{$X_1$}}
%     }
%   }
%   child {
%     node[and]{}
%     child {
%       node[var] {\rotatebox{-90}{$X_3$}}
%     }
%     child {
%       node[var] {\rotatebox{-90}{$X_2$}}
%     }
%     child {
%       node[var] {\rotatebox{-90}{$X_1$}}
%     }
%   }
%   child {
%     node[and]{}
%     child {
%       node[var] {\rotatebox{-90}{$X_4$}}
%     }
%     child {
%       node[var] {\rotatebox{-90}{$X_3$}}
%     }
%     child {
%       node[var] {\rotatebox{-90}{$X_1$}}
%     }
%   }
%   child {
%     node[and]{}
%     child {
%       node[var] {\rotatebox{-90}{$X_3$}}
%     }
%     child {
%       node[var] {\rotatebox{-90}{$X_2$}}
%     }
%     child {
%       node[var] {\rotatebox{-90}{$X_1$}}
%     }
%   };
% \end{tikzpicture}

\begin{tikzpicture}[
    circuit logic US,
    tiny circuit symbols,
    level 1/.style={sibling distance=5*18pt},
    level 2/.style={sibling distance=18pt},
    ]
  \node[or]{\rotatebox{-90}{Y}}
  % Subsets of size 3
  child { node[and] {} 
    child { node[var] {\rotatebox{-90}{$X_1$}} }
    child { node[var] {\rotatebox{-90}{$X_2$}} }
    child { node[var] {\rotatebox{-90}{$X_3$}} }
  }
  child { node[and] {} 
    child { node[var] {\rotatebox{-90}{$X_1$}} }
    child { node[var] {\rotatebox{-90}{$X_2$}} }
    child { node[var] {\rotatebox{-90}{$X_4$}} }
  }
  child { node[and] {} 
    child { node[var] {\rotatebox{-90}{$X_1$}} }
    child { node[var] {\rotatebox{-90}{$X_2$}} }
    child { node[var] {\rotatebox{-90}{$X_5$}} }
  }
  child { node[and] {} 
    child { node[var] {\rotatebox{-90}{$X_1$}} }
    child { node[var] {\rotatebox{-90}{$X_3$}} }
    child { node[var] {\rotatebox{-90}{$X_4$}} }
  }
  child { node[and] {} 
    child { node[var] {\rotatebox{-90}{$X_1$}} }
    child { node[var] {\rotatebox{-90}{$X_3$}} }
    child { node[var] {\rotatebox{-90}{$X_5$}} }
  }
  child { node[and] {} 
    child { node[var] {\rotatebox{-90}{$X_1$}} }
    child { node[var] {\rotatebox{-90}{$X_4$}} }
    child { node[var] {\rotatebox{-90}{$X_5$}} }
  }
  child { node[and] {} 
    child { node[var] {\rotatebox{-90}{$X_2$}} }
    child { node[var] {\rotatebox{-90}{$X_3$}} }
    child { node[var] {\rotatebox{-90}{$X_4$}} }
  }
  child { node[and] {} 
    child { node[var] {\rotatebox{-90}{$X_2$}} }
    child { node[var] {\rotatebox{-90}{$X_3$}} }
    child { node[var] {\rotatebox{-90}{$X_5$}} }
  }
  child { node[and] {} 
    child { node[var] {\rotatebox{-90}{$X_2$}} }
    child { node[var] {\rotatebox{-90}{$X_4$}} }
    child { node[var] {\rotatebox{-90}{$X_5$}} }
  }
  child { node[and] {} 
    child { node[var] {\rotatebox{-90}{$X_3$}} }
    child { node[var] {\rotatebox{-90}{$X_4$}} }
    child { node[var] {\rotatebox{-90}{$X_5$}} }
  }
  % Subsets of size 4
  child { node[and] {} 
    child { node[var] {\rotatebox{-90}{$X_1$}} }
    child { node[var] {\rotatebox{-90}{$X_2$}} }
    child { node[var] {\rotatebox{-90}{$X_3$}} }
    child { node[var] {\rotatebox{-90}{$X_4$}} }
  }
  child { node[and] {} 
    child { node[var] {\rotatebox{-90}{$X_1$}} }
    child { node[var] {\rotatebox{-90}{$X_2$}} }
    child { node[var] {\rotatebox{-90}{$X_3$}} }
    child { node[var] {\rotatebox{-90}{$X_5$}} }
  }
  child { node[and] {} 
    child { node[var] {\rotatebox{-90}{$X_1$}} }
    child { node[var] {\rotatebox{-90}{$X_2$}} }
    child { node[var] {\rotatebox{-90}{$X_4$}} }
    child { node[var] {\rotatebox{-90}{$X_5$}} }
  }
  child { node[and] {} 
    child { node[var] {\rotatebox{-90}{$X_1$}} }
    child { node[var] {\rotatebox{-90}{$X_3$}} }
    child { node[var] {\rotatebox{-90}{$X_4$}} }
    child { node[var] {\rotatebox{-90}{$X_5$}} }
  }
  child { node[and] {} 
    child { node[var] {\rotatebox{-90}{$X_2$}} }
    child { node[var] {\rotatebox{-90}{$X_3$}} }
    child { node[var] {\rotatebox{-90}{$X_4$}} }
    child { node[var] {\rotatebox{-90}{$X_5$}} }
  }
  % Subset of size 5
  child { node[and] {} 
    child { node[var] {\rotatebox{-90}{$X_1$}} }
    child { node[var] {\rotatebox{-90}{$X_2$}} }
    child { node[var] {\rotatebox{-90}{$X_3$}} }
    child { node[var] {\rotatebox{-90}{$X_4$}} }
    child { node[var] {\rotatebox{-90}{$X_5$}} }
  };
\end{tikzpicture}
    }
    \caption{Caption}
    \label{fig:example-3of5-voter-tree}
\end{figure}

Since ``at least 3 of 5'' means any subset of size 3, 4, or 5 suffices, an explicit OR-of-ANDs expansion has the following terms:
\begin{equation}
\begin{aligned}
Y \;=\;&
\underbrace{
  \resizebox{.90\hsize}{!}{$
    X_1 X_2 X_3 \;\lor\; X_1 X_2 X_4 \;\lor\; X_1 X_2 X_5 \;\lor\; X_1 X_3 X_4 \;\lor\; X_1 X_3 X_5 \;\lor\; X_1 X_4 X_5 \;\lor\; X_2 X_3 X_4 \;\lor\; X_2 X_3 X_5 \;\lor\; X_2 X_4 X_5 \;\lor\; X_3 X_4 X_5
  $}
}_{\text{subsets of size 3}}
\\
&\;\;\lor\;\;
\underbrace{
  \resizebox{.80\hsize}{!}{$
    X_1 X_2 X_3 X_4 \;\lor\; X_1 X_2 X_3 X_5 \;\lor\; X_1 X_2 X_4 X_5 \;\lor\; X_1 X_3 X_4 X_5 \;\lor\; X_2 X_3 X_4 X_5
  $}
}_{\text{subsets of size 4}}
\\
&\;\;\lor\;\;
\underbrace{X_1 X_2 X_3 X_4 X_5}_{\text{subset of size 5}}.
\end{aligned}
\end{equation}

Assume that each \(X_i\) is a Bernoulli random variable taking the value 1 with probability \(p_i\), independently of the others. Then
\[
p_i 
\;=\;
P\bigl(X_i = 1\bigr),
\quad
i = 1,2,3,4,5.
\]
We seek 
\(\displaystyle P(Y=1)\), i.e., the probability that at least three of the \(X_i\) are 1.  A convenient way to write this is to sum, for \(j=3,4,5\), the probabilities that exactly \(j\) of the inputs are 1:
\[
P\bigl(Y=1\bigr)
\;=\;
\sum_{j=3}^{5}
\;\sum_{\substack{S\,\subseteq\,\{1,2,3,4,5\}\\|S|=j}}
\;\Bigl(\prod_{i\in S} p_i\Bigr)\;
\Bigl(\prod_{i\notin S} (1 - p_i)\Bigr).
\]
Exactly three of the five inputs are 1 (there are \(\binom{5}{3} = 10\) such terms):
\[
\begin{aligned}
&\,p_1 p_2 p_3\,(1\!-\!p_4)\,(1\!-\!p_5)\;+\;
p_1 p_2 p_4\,(1\!-\!p_3)\,(1\!-\!p_5)\;+\;
p_1 p_2 p_5\,(1\!-\!p_3)\,(1\!-\!p_4)\;+\;
\\
&\;p_1 p_3 p_4\,(1\!-\!p_2)\,(1\!-\!p_5)\;+\;
p_1 p_3 p_5\,(1\!-\!p_2)\,(1\!-\!p_4)\;+\;
p_1 p_4 p_5\,(1\!-\!p_2)\,(1\!-\!p_3)\;+\;
\\
&\;p_2 p_3 p_4\,(1\!-\!p_1)\,(1\!-\!p_5)\;+\;
p_2 p_3 p_5\,(1\!-\!p_1)\,(1\!-\!p_4)\;+\;
p_2 p_4 p_5\,(1\!-\!p_1)\,(1\!-\!p_3)\;+\;
\\
&\;p_3 p_4 p_5\,(1\!-\!p_1)\,(1\!-\!p_2).
\end{aligned}
\]
Exactly four of the five inputs are 1 (there are \(\binom{5}{4}=5\) such terms):
\[
\begin{aligned}
&p_1 p_2 p_3 p_4\,(1\!-\!p_5)
\;+\;
p_1 p_2 p_3 p_5\,(1\!-\!p_4)
\;+\;
p_1 p_2 p_4 p_5\,(1\!-\!p_3)
\\
&\quad+\;
p_1 p_3 p_4 p_5\,(1\!-\!p_2)
\;+\;
p_2 p_3 p_4 p_5\,(1\!-\!p_1).
\end{aligned}
\]
All five inputs are 1 (there is \(\binom{5}{5}=1\) such term):
\[
p_1 p_2 p_3 p_4 p_5.
\]

Summing all of the above terms yields \(P(Y=1)\), the fully expanded probability of a 3-of-5 vote under independence.

\subsection{Complexity of the Inclusion-Exclusion Approach}
Observe that the union in Eq.~\eqref{eq:union_inclusion_exclusion} comprises \(M\) events, where
\[
M \;=\;\sum_{j=k}^{n} \binom{n}{j}.
\]
The inclusion-exclusion principle for a union of \(M\) events involves sums of intersections of size \(r\), for \(r=1,\dots,M\). The total number of terms is
\[
\sum_{r=1}^{M} \binom{M}{r}
\;=\;
2^{M}-1,
\]
which implies a worst-case computational complexity on the order of
\[
\mathcal{O}\bigl(2^{M}\bigr).
\]
Moreover, for \(k\approx n/2\), the binomial coefficients \(\binom{n}{k}\) become largest, so \(M\) can itself grow exponentially in \(n\). Consequently, the inclusion-exclusion expansion may require up to \(\mathcal{O}\bigl(2^{2^{n}}\bigr)\) operations for moderately large \(n\).
