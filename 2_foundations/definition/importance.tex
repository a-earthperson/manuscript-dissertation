\section{Computing Importance Measures}
\label{sec:foundations_importance}
Importance measures quantify the contribution of basic events to system reliability or risk. They provide insights into which components or events are most critical to system performance, guiding resource allocation for maintenance, design improvements, and risk management. \acrshort{pra} tools typically calculate the following importance measures to support comprehensive reliability analysis.

\subsection{Conditional Importance Measure (CIM)}
The \emph{\acrfull{cim}}  of a basic event $i$ measures the probability that event $i$ contributes to system failure, given that the system has failed. It represents the fraction of system failures that involve the basic event.

\[
\text{CIM}_i = \frac{P(i \text{ contributes to system failure} \mid \text{system failure})}{P(\text{system failure})} = \frac{P(i \text{ critical})}{P(\text{system failure})}
\]

This is calculated by identifying minimal cut sets containing the basic event and determining the proportion of failure probability attributed to these cut sets.

\subsection{Marginal/Birnbaum Importance Measure (MIM)}
The \emph{\acrfull{mi}} measures the rate of change in system reliability with respect to the reliability of component $i$. It quantifies how sensitive the system failure probability is to changes in the failure probability of the component.

\[
\text{MIM}_i = \frac{\partial P(\text{system failure})}{\partial P(i \text{ fails})} = P(\text{system fails} \mid i \text{ fails}) - P(\text{system fails} \mid i \text{ succeeds})
\]

This is computed by evaluating the difference in system failure probability when the component is assumed failed versus successful.

\subsection{Potential Importance Measure (PIM)}
The \emph{\acrfull{pim}} the maximum possible reduction in system failure probability that could be achieved by improving component $i$. It indicates the potential benefit of perfect reliability for a component.

\[
\text{PIM}_i = P(\text{system failure}) - P(\text{system failure} \mid i \text{ succeeds}) = P(\text{system failure}) \times (1 - \frac{1}{\text{RRW}_i})
\]

This is calculated using the system failure probability under current conditions compared to the system failure probability when the component is perfectly reliable.

\subsection{Diagnostic Importance Measure (DIM)}
The \emph{\acrfull{di}} measures the probability that component $i$ has failed, given that the system has failed. It is useful for fault diagnosis and identifying likely causes of system failure.

\[
\text{DI}_i = \frac{P(i \text{ fails} \mid \text{system fails})}{P(i \text{ fails})} = \frac{P(i \text{ fails} \cap \text{system fails})}{P(i \text{ fails}) \times P(\text{system fails})}
\]

This is calculated by determining the conditional probability of component failure given system failure, normalized by the component's failure probability.

\subsection{Criticality Importance Measure (CRI)}
The \emph{\acrfull{cri}} combines the Marginal Importance with the probability of component failure. It measures the contribution of component $i$ to the overall system failure probability.

\[
\text{CRI}_i = \text{MI}_i \times \frac{P(i \text{ fails})}{P(\text{system fails})} = \frac{P(i \text{ fails}) \times [P(\text{system fails} \mid i \text{ fails}) - P(\text{system fails} \mid i \text{ succeeds})]}{P(\text{system fails})}
\]

This is computed by multiplying the Marginal Importance by the ratio of component failure probability to system failure probability.

\subsection{Risk Achievement Worth (RAW)}
The \emph{\acrfull{raw}} measures the factor by which system failure probability increases when component $i$ is assumed to have failed. It indicates the importance of maintaining the current reliability of the component.

\[
\text{RAW}_i = \frac{P(\text{system fails} \mid i \text{ fails})}{P(\text{system fails})}
\]

RAW is computed by comparing the system failure probability when the component is assumed failed to the baseline system failure probability.

\subsection{Risk Reduction Worth (RRW)}
The \emph{\acrfull{rrw}} measures the factor by which system failure probability decreases when component $i$ is assumed perfectly reliable. It indicates the potential value of improving component reliability.

\[
\text{RRW}_i = \frac{P(\text{system fails})}{P(\text{system fails} \mid i \text{ succeeds})}
\]

RRW is calculated by comparing the baseline system failure probability to the system failure probability when the component is assumed perfectly reliable.