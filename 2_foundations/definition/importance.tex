\section{Importance Measures for Basic Events}
\label{sec:foundations_importance}
Reliability and \acrshort{pra} studies rely on 
\emph{importance measures} to quantify how individual basic events 
contribute to an overall risk metric~$R$ (such as a top‐event probability
in a mission time window or a frequency measure like \acrshort{cdf} or
\acrfull{lerf}).
These measures support ranking, screening, design improvement, and
configuration‐risk management.  This section formalizes the measures
most widely used in \acrshort{pra}/\acrshort{qra}, adopting notation and
sign conventions consistent with the \acrshort{nasa} \acrshort{pra} Practitioners' Guide ~\cite{smith_probabilistic_2012} and
\acrshort{us} \acrshort{nrc} risk‐informed guidance.

\subsection*{Notation}

\begin{itemize}
  \item $q_i$ – failure/unavailability probability of basic event~$i$.
  \item $x_i$ – binary state indicator of event~$i$ ($x_i=1$: failed, $x_i=0$: succeeded).
  \item $R$ – baseline risk metric (probability or frequency).
  \item $R\mid x_i=1$ – risk when event~$i$ is forced failed.
  \item $R\mid x_i=0$ – risk when event~$i$ is forced successful.
  \item $\mathcal{MCS}$ – union of all minimal cut sets (failure space).
  \item $\mathcal{MCS}_i$ – union of minimal cut sets that contain event~$i$.
\end{itemize}

Throughout, forcing a basic event applies the Boolean absorption
$x_i=1\!\rightarrow\!\text{true}$ or $x_i=0\!\rightarrow\!\text{false}$ in
the fault‐tree logic before re‐quantification.

% --------------------------------------------------------------------
\subsection{Fussell–Vesely Importance (FV)}
% --------------------------------------------------------------------

\begin{equation}
  \label{eq:fv}
  \mathrm{FV}_i 
  = \frac{\Pr(\cup\,\mathcal{MCS}_i)}{\Pr(\cup\,\mathcal{MCS})}
  = \frac{R - R\mid x_i=0}{R}.
\end{equation}

Equation~\eqref{eq:fv} expresses the fractional contribution of basic
event~$i$ to the current risk.  High \acrshort{fv} values indicate that a
large proportion of current risk flows through cut sets containing~$i$;
it is therefore the workhorse metric for contributor ranking and
screening.

% --------------------------------------------------------------------
\subsection{Risk Reduction Worth (RRW)}
% --------------------------------------------------------------------

\begin{equation}
  \label{eq:rrw}
  \mathrm{RRW}_i = \frac{R}{R\mid x_i=0}, \qquad
  \Delta R^{-}_i = R - R\mid x_i=0.
\end{equation}

A large \acrshort{rrw} signals substantial potential benefit from
reliability or availability upgrades of component~$i$.  The interval
form~$\Delta R^{-}_i$ is useful when an absolute (rather than relative)
risk decrement is required.

% --------------------------------------------------------------------
\subsection{Risk Achievement Worth (RAW)}
% --------------------------------------------------------------------

\begin{equation}
  \label{eq:raw}
  \mathrm{RAW}_i = \frac{R\mid x_i=1}{R}, \qquad
  \Delta R^{+}_i = R\mid x_i=1 - R.
\end{equation}

\acrshort{raw} asks how much risk would be \emph{achieved} (increased) if
component~$i$ were failed.  It is central to configuration‐risk
management; industry practice often screens for $\mathrm{RAW}_i>2$ when
identifying safety‐significant items.

% --------------------------------------------------------------------
\subsection{Birnbaum (Marginal) Importance (MIM)}
% --------------------------------------------------------------------

\begin{equation}
  \label{eq:birnbaum}
  I_{\mathrm{B}}(i) = \frac{\partial R}{\partial q_i}
  \approx R\mid x_i=1 - R\mid x_i=0,\quad \text{(linear‐risk models)}
\end{equation}

Birnbaum importance is a local sensitivity: it measures the structural
leverage of~$i$ on~$R$.  Because it is not scaled by~$q_i$, a very
reliable yet pivotal component can have a high $I_{\mathrm{B}}$.

% --------------------------------------------------------------------
\subsection{Differential Importance Measure (DIM)}
% --------------------------------------------------------------------
For a contemplated small change vector $\{\mathrm{d}q_j\}$,
\begin{equation}
  \label{eq:dim}
  \mathrm{DIM}_i = \frac{\displaystyle \frac{\partial R}{\partial q_i}\,\mathrm{d}q_i}
                        {\displaystyle \sum_j \frac{\partial R}{\partial q_j}\,\mathrm{d}q_j}.
\end{equation}
The \acrshort{dim}s sum to~1 and allocate the total risk change across
parameters, which can be useful when many small improvements are pursued
simultaneously.

% --------------------------------------------------------------------
\subsection{Criticality Importance (\acrshort{cri})}
% --------------------------------------------------------------------
\begin{equation}
  \label{eq:criticality}
  \mathrm{CRI}_i = I_{\mathrm{B}}(i)\,\frac{q_i}{R}
  = \frac{q_i\,[R\mid x_i=1 - R\mid x_i=0]}{R}.
\end{equation}
\acrshort{cri} gives the posterior probability that event~$i$ has both
failed and is \emph{critical} when the top event occurs, supporting
post‐event diagnostics and maintenance prioritization.

% --------------------------------------------------------------------
\subsection{Diagnostic Importance Factor (\acrshort{di})}
% --------------------------------------------------------------------
\begin{equation}
  \label{eq:diagnostic}
  \mathrm{DI}_i = \Pr(x_i=1 \mid \text{Top}) = \frac{q_i\,R\mid x_i=1}{R}.
\end{equation}
Unlike \acrshort{cri}, \acrshort{di} does not require that~$i$ be
pivotal, only that it has failed.  It thus quantifies the probability
that~$i$ is the culprit in need of inspection once the system has
failed.

% --------------------------------------------------------------------
\subsection{Improvement Potential (IP)}
% --------------------------------------------------------------------
\begin{equation}
  \label{eq:ip}
  \mathrm{IP}_i = R - R\mid x_i=0 = R\,\mathrm{FV}_i.
\end{equation}
\acrshort{pim} expresses the absolute risk points removable by perfecting
component~$i$.  When paired with a cost metric, it guides benefit‐per‐dollar prioritization in maintenance optimization.

% --------------------------------------------------------------------
\subsection{Prevention Worth (\acrshort{pw})}
% --------------------------------------------------------------------
In the success domain (path‐set analysis), \acrshort{pw} formalizes how much the overall \emph{mission success} probability improves when basic event~$i$ is guaranteed to succeed.  Let
\begin{equation}
  \label{eq:success_prob}
  S \;=\; 1 - R \;=\; \Pr\!\bigl(\cap\,\mathcal{MPS}\bigr),
\end{equation}
where $S$ is the baseline probability of reaching the top success state and $\mathcal{MPS}$ denotes the union of all minimal path sets.

Forcing $x_i=0$ in the logic (i.e., assuring component~$i$ succeeds) yields an updated success probability $S\mid x_i=0 = 1 - R\mid x_i=0$.  The \emph{Prevention Worth} of~$i$ is then
\begin{equation}
  \label{eq:pw}
  \mathrm{PW}_i 
  \;=\;
  \frac{S\mid x_i=0}{S}
  \;=\;
  \frac{1 - R\mid x_i=0}{1 - R},
  \qquad
  \Delta S^{+}_i = S\mid x_i=0 - S.
\end{equation}
A value $\mathrm{PW}_i>1$ signals that assuring the success of component~$i$ produces a proportional uplift in overall mission success, while the interval form~$\Delta S^{+}_i$ gives the absolute success points gained.  These metrics support availability optimization and design‐for‐success studies in contexts where objectives are framed in terms of maximizing~$S$ rather than minimizing~$R$.
