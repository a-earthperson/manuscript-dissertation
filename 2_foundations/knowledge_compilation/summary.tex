% table summarizing equivalent forms, columns for:

% 1. equivalence: bijective, etc.
% 2. canonical: uniquenuess, etc.
% 3. uses: bdd for prob, nnf for mocus, anf removes negations, etc.
% 4. relative size on conversion
% 5. conversion time complexity
% 6. real-world performance constraints for conversion

% The semantics of the risk model-language have been developed for succinct representation of risk-reliability. That is exactly †he intent - = a language should serve the needs of the represented domain. Howecver
% Knowing what problems can be solved efficiently.

The traditional PRA modeling language—with its separate constructs for Initiating Events, Event Trees, and Fault Trees—was devised for human readability and incremental model construction.  Those distinctions help engineers reason about failure propagation and refine a model’s granularity, yet they also fragment information across heterogeneous data structures.  For small systems this heterogeneity is innocuous; for today’s models containing hundreds of event trees and thousands of fault trees it becomes a barrier to scalable analysis, impeding both exact probability calculations and qualitative queries such as minimal-cut-set enumeration.

To overcome these limitations we advocate translating the entire model into a single probabilistic directed acyclic graph (PDAG).  The transformation is loss-free, deterministic, and yields a representation that is simultaneously human-verifiable and machine-amenable.  Once in PDAG form the model enjoys three key benefits: (i) a uniform graph data structure that supports high-performance algorithms for probability estimation and structural queries, (ii) compatibility with knowledge-compilation techniques that enable selective normal-form transformations, and (iii) the ability to accommodate incremental design changes through lightweight graph updates.

The remainder of this chapter explains the PDAG construction, illustrates the mapping on the running example of Fig.~\ref{fig:et_ft_example}, and shows how classical PRA operations can be re-expressed as graph traversals or as queries over a propositional knowledge base.  This perspective not only clarifies the formal semantics of PRA models but also lays the groundwork for the data-parallel Monte-Carlo methods developed in later chapters.

\begin{figure}[p]
    \centering
    \includesvg[width=1\textwidth]{figs/pdag/pra-model.svg}
    \caption{A working example: Starting with an initiating event (I), an event tree (ET) with three linked fault trees (FT), and shared basic events (FT), and five end states (ES).}
    \label{fig:et_ft_example}
\end{figure}

In this section, we will show how these requirements can be achieved by viewing risk models as \acrfull{pdag}. We show how \acrshort{pdag} model maps on \acrshort{pdag} and how standard \acrshort{pra} methods can be viewed as operations on graphs. 
Furthermore, we show how \acrshort{pdag}s can be viewed as collections of propositional logic statements --- knowledge bases.
This view provides fundamental mathematical rigor to \acrshort{pra} formalism, by casting PRA methods as queries over and transformation of aforementioned knowledge bases.
Not only this formalism provides strong computational bounds and guarantees for the algorithms of interest, but it also explicitly separates the preparation and analysis steps of \acrshort{pra} model, allowing for efficient separation and querying.

% these are just suggestions! do as you please.